% This is a default-selection of plugins that are used widely in this repo.

\documentclass[a4paper,10pt,fleqn]{article}
\usepackage[utf8]{inputenc}

% deutsche Trennmuster etc.
\usepackage[ngerman]{babel}
\usepackage[T1]{fontenc}

% mathematical simbols and fonts
\usepackage{mathtools} 
\usepackage{amssymb}
\usepackage{amsmath}
\usepackage{ntheorem}
\usepackage{polynom}
\usepackage{marvosym}
\usepackage{tabu}
\renewcommand*{\bmod}{\mathbin{\%}}
\everymath{\displaystyle}

\usepackage{multicol}
\usepackage{color}
\usepackage[usenames,dvipsnames]{xcolor}
\setlength{\columnsep}{1cm}
\setlength{\columnseprule}{0.25pt}
\def\columnseprulecolor{\color{gray}}
\usepackage{hyperref}

\usepackage[margin=1.5cm]{geometry}
\usepackage{graphicx}
\usepackage{pgfplots}
\pgfplotsset{compat=1.10}

%Code higlighting

\usepackage{minted}

% make lists more compact:
\newlength{\wideitemsep}
\setlength{\wideitemsep}{.5\itemsep}
\addtolength{\wideitemsep}{-5pt}
\let\olditem\item
\renewcommand{\item}{\setlength{\itemsep}{\wideitemsep}\olditem}
\renewcommand{\arraystretch}{1.25}

\title{Zusammenfassung CN2}
\author{Fabian Hauser}
 
\begin{document}
\maketitle

\section{Varia}
10 Seiten A4 Zusammenfassung

\section{Telefonie}

\begin{description}
\item[Data Plane] \hfill \\
	Forward IP-Packets. Z.B. STP, VLANs
\item[Control Plane] \hfill \\
	Routing Protocols, Overlay technologies (MPLS), z.B: Forwarding of Ethernet Frames
\end{description}


\subsection{G.711 / PCM}
 256 Audio-Steps using 8 Bits, Sample 8000/sec (nyquist frequency) = DS0, 64Kbps


\subsection{ISDN}
Braucht NT Wandler.

\begin{description}
\item[S-Bus] \hfill \\
	 Digitaler Bus im Haus; Data Plane.
\item[D-Kanal] \hfill \\
	Synchronisierungs-Kanal zur Dorfzentrale
\item[PBX] \hfill \\
	Die Dorfzentrale; Übersetzt Rufnummer-Anfragen (Wählton) zu weiteren Zentralen

\end{description}

\subsection{SS7}

Ähnlich BGP ist das SS7 Routing Protokoll die Verbindung zwischen allen Telefonzentralen.
	
\subsection{Voip}
Läuft über Router; muss von Telefonnummer eine IP-Addresse "extrahieren".
	
\subsubsection{The Connection / Bearer Plane}
Switching Logic: Connection Negotiation, Transcoding

Mittels z.B. SIP mit CCP Verbunden.

\subsubsection{The Call Control Plane: SIP as a Example}
Call Logic: Signalling \& Call control
Logic / Services

Mittels z.B. TAPI / INAPI mit Services Plane verbunden.

Distributed call processing: (SIP, H.323), Centralized cal process (SCCP, MGCP, MEHACO, H.248)

Inititation von Verbindungen

\subsubsection{The Services Plane}
Service Logic: IN Service Logic, AAA, Address Resolution


\subsection{Session Description Protocol SDP}
Manage the Lines respectivly connection properties

\begin{description}
\item INVITE \hfill \\
	codecs support etc.
\item REGISTER
\item BYE
\item ACK 
\item CANCEL
\item OPTIONS
\end{description}

Mediaübertragung über RTP.

\subsection{SIP}

\begin{itemize}
	\item User location
	\item User availability
	\item User capabilies
	\item Session setup
	\item Session management
\end{itemize}

Makes use of:
\begin{itemize}
	\item HTTP
	\item SDP
	\item RTP / RTSP
	\item URI / URL
	\item DHCP / DNS
	\item MIME
	\item TLS / IPSec
\end{itemize}

\begin{description}
	\item Registrar Server \hfill \\
		Datase of IP / Numbers matching of the User Agent
	\item Proxy \hfill \\
		Routing of SIP signalling messages
	\item Redirect Server \hfill \\
		Redirects von Anrufen
\end{description}

\section{Routing}

Layer 3, dynamisch oder statisch

\subsection{Routing Hierarchy}

Separiert in \emph{Autonomous Systems} (AS). Zwischen den AS wird mit \emph{Exterior Routing Protocols} (EGP) benutzt, üblicherweise BGP.

Innerhalb der AS wird ein Interior routing Protocol (IGP) verwendet.

\subsection{Distance Vector}

Es wird jeweils die ganze Tabelle zurückgegeben.

\subsubsection{Split Horizon}

Der Router, welcher die Route mitgeteilt hat, wird nicht "wiederholt".

\subsection{Link State Protocol}

\begin{enumerate}
	\item Ermittlung Nachbarn mit Hello Protocol
	\item Kenntnisse Nachbarn weitergeben. (\emph{Link State Advertisments})
	\item \emph{Shortest Path First} nach kenntniss von ganzer Datenbank
\end{enumerate}

Es wir jeweils nur ein Single Entry durchgegeben, mit dem Inhalt (RouterID, NeighborID, Cost)

\subsubsection{Shortest Path Algorithm}

\begin{enumerate}
\item %TODO	
\end{enumerate}


\subsection{OSPF}

\begin{itemize}
	\item Secure (Signing)
	\item Classless
	\item Max 50 client per Area
\end{itemize}


\subsection{BGP}

Border Gateway Protocoll

\subsubsection{Autonomouse Systems}

1-65535 unique
private AS pool from 64512 to 65535 (similar to private IPs)
RFC 6793 introduces 32 bit AS numbers
(y.x), each x/y = 16bits
BGP4, is used to interconnect AS

\subsubsection{Connection}

\paragraph{Home} \hfill \\

Access to ISP; default route

\paragraph{Multi-Homed} \hfill \\
Connected with muliple AS Provider or multiple Links to the same Provider

\begin{description}
	\item[Transit AS] \hfill \\
	Can be used for transit traffic by other providers
	\item[Nontransit AS] \hfill \\
	Advertises only its own routes to both providers
\end{description}

\subsubsection{Protocol}

The less hops the better

Communicates in which AS are which IP Ranges.

BGP4 supports CIDR and route aggregation

BGP is a Path vector protocol

TCP Port 179

\begin{enumerate}
	\item OPEN MESSAGE: Exchange AS, router ID, holdtime, Capability negotiation
	\item NOTIFICATION (Example: ''peer in wrong AS'')
	\item KEEPALIVE when no updates
	\item UPDATE (incremental)
\end{enumerate}


Netze werden explizit Withdrwan

Attributes: Pfadlängen, Kosten etc.

Prefixes: Neuer Weg etc.

\subsubsection{Types}
(\emph{Nicht} zum auswendig-lernen)
\begin{description}
	\item[1: ORIGIN]
	\item[2: AS-PATH]
	\item[3: NEXT-HOP]
	\item[4: MED] Multi Exit Discriminator (Reccomandations Traffic ports)
	\item[5: LOCAL\_PREF] Local preferences
\end{description}

\includegraphics[width=0.5\linewidth]{img/bgp_process_model.png}

\subsubsection{Peering vs. Transit}

Transit: Kostet, quasi default gateway wo via peering nicht abgedeckt.
Peering: Lokale Provider verknüpfen. (üblicherweise Gratis).

\paragraph{Peering}
Multi-lateral Peering: Zentrale Einheit welche das Peering organisiert (Zentraler Swtich)

Bi-lateral Peering: Direktes Peering mit anderem Provider.


\section{IP}

\subsection{Multicast}

Single Sender, Multiple Receivers. Darum UDP mit entsprechenden Nachteilen.


Multicast Adressen sind zwischen 224.0.0.0 bis 239.255.255.255 bei IPv4.

Link Local Adrdres: 224.0.0.0 - 224.0.0.255
Global Address Range: 224.0.1.0-238.255.255.255
Administratively Scoped Address Range (Privately used addresses within private multicast domain): 239.0.0.0 - 239.255.255.255



\subsubsection{MAC Layer 2}
	Es gibt auch Multicast-Mac-Adressen 00-00-00 bis 01-00-5e-7f-ff-ff.
	Es werden jeweils die letzten 23-Bit der IP an 01-00-5e- angegängt.
	
	Von der IPv4 Adresse werden die ersten neun Bits abgehackt.

\subsubsection{IGMP}

Internet Group Management Protocol

Local LAN to Router, wished group memberships.

IGMPv3: Join / Leave to/from explicit sources and broadcast addresses.

\subsubsection{Wide Area}
üblicherweise PIM

\subsubsection{Multicast Distribution Trees}

Zur verhinderung von Loops.

Shortest Path oder Source Tree bei einfachen Trees
-> More memory n(S x G) but optimal paths,


Shared Tree (PIM-Sparse Mode): PIM Rendezvous Point 
-> Less memory n(G) but no optimal paths. The client can afterwards directly connect tothe source

\emph{(source,dest)}
Z.B: \emph{(*, 224.20.10.1)}

\subsubsection{RPF}

Reverse Path Forwarding

Es wird von der Destionation zur Source einen Weg aufgebaut.

Jeweils zum nächsten Weg zur Source (im Zweifelsfall der ''letzte'' hop richting root.

Dense Mode: ''Push'' model: traffic floded through net. Can be unsubscribed with ''Prune''

Sprase-mode: ''Pull'' model: traffic on subscription. Client connects to Randevouz Point, The Source also Informed the RP.

\subsubsection{IGMPv1-v2 Snooping}

Nur Geräte, welche sich per IGMP subscriben, kriegen vom Switch die entsprechenden Daten.



\section{Wide Area Networks WAN}

Vernetzung von Standorten.

\subsection{Leased line}

Private line between endpoints.

N x 64kbps

Provider does usually Multiplexing.

\paragraph{E1 Framing I.}

32 Slots * 8bits

Möglich, mehre Kanäle per Multiplexing zusammenzuhängen.

\subsection{SDM / TDH}?

\subsection{Synchronous Digital Hierarchy (SDH)}

''Container'' gehen immer im Ring herum. Es gibt eine Spur in jeweils jede Richtung.

Ein Add/Drop Multiplexer Fügt/Entfernt Pakete aus dem Container.

\subsection{PPP}

Layer 2

Point to Point Protocoll

(Sehr ähnlich Ethernet)

Regelt IP Vergabe, Link Configuration, Quality, Error Detection

Authentication, Callback, compression, Multilink

\subsection{Frame Relay}

\begin{itemize}
	\item Layer 2
	\item Macht ein Backbone taugliches ''Ethernet'' im WAN.
	\item Mehrere logische Verbindungen über eine physische Leitung. Dadurch: Bandbreite geshared.
	\item Packet Orientiert
	\item PVC (Permanent Virtual Circuits)
	\item SVC (Switched Virtual Circuits)
	\item Virtuelle Verbindungen werden als: DLCI (Data Link Connection Identifier) bezeichnet.
	\item Es hat 1000 mögliche Adressen
	\item Kennt keine Redundanz etc.
	\item QoS \begin{itemize}
		\item Treffic shaper beim Endpunkt (Steuerung vom Kunden).
		\item Traffic Policer (Harte forcierung vom Provider)
		\item Congestion Flags from Network (FECN, BECN, DE-bit)
	\end{itemize}
	\[
		CIR = \frac{Bc}{Ts}
	\]
\end{itemize}

Addressen haben eine Lokale Signifikatent (Die Nummer beschreibt pro Ziffer den zu wählenden Weg.

\subsubsection{Begriffe}
	%TODO: insert img/frame_relay_...


\subsection{ATM}

\begin{itemize}
	\item Cell based switching.
	\item Wird u.a. von ADSL und UMTS verwendet.
	\item Realtiv schnell (Immer Ethernet etwas voraus im Speed).
	\item Ermöglicht leichtes integrieren von QoS
	\item Schnelles switching da in Hardware realisiert
	\item Erlaubt CBR (Constant Bit Rate)0
\end{itemize}

\subsection{MPLS: Multiprotocol Label Switching}

Vorteile: Skalierung, Plug'n'play, just like an IP net


Features: QoS, Traffic Engineering, OAM/MIBs

Usually: Service Provider, Datacenter Provider

\subsubsection{Working}

Label based routing (the router only knows, where to forward a specific label / interface to.

\begin{description}
	\item[P (Provider) router = label switching router = core router (LSR)] \hfill \\
		Switches MPLS-labeled packets
		Runs an IGP and LDP
	\item[PE (Provider Edge) router = edge router (LSR)] \hfill \\
		Imposes and removes MPLS labels
		Runs an IGP, LDP and MP-BGP
	\item[CE (Customer Edge) router] \hfill \\
		Connects customer network to MPLS network
	\item[Route-Target] \hfill \\
		64 bits identifying routers that should receive the route
	\item[Route Distinguisher] \hfill \\
		Attribute of each route used to uniquely identify prefixes among VPNs (64 bits)
		VRF based (not VPN based)
	\item[VPN-IPv4 addresses] \hfill \\
		Address including the 64 bits Route Distinguisher and the 32 bits IP address
	\item[VRF] \hfill \\
		Virtual routers: Routing of \emph{(TAG)IP}
	\item[FEC] \hfill \\
		is a group of IP packets which are forwarded in the same manner
	\item[LSR] \hfill \\
		Label Switch Router
	\item[LIB] \hfill \\
		Label Information Base (DB of labels)
	\item[GRE] \hfill \\
		IP in IP Verschachteln (unverschlüsselt).
\end{description}

Layering with multiple Labels are possible.


\subsubsection{LDP}

Label Distribution Protocol.

Ähnliches Protokoll: RSVP (reserviert Ressourcen)

\subsubsection{L2TPv3}

Only IP Core is neccessary; no MPLS to allow e.g. Ethernet Virtual Wire Service.

\subsubsection{AToM: Any Technology over MPLS}

Buils on MPLS Core; Wire Connection

Hat VC-Label (VLAN).; Macht Layer 2 über längere Distanzen.

\subsubsection{VPLS}

Virtual Private LAN Service:

MPLS-Core.

Letzte zwei sind heute MPLS Layer-2 VPNs.

Verhinderung von Loop: Spanning Tree wird durch Split-Horizon ersetzt.


\subsubsection{QoS \& Traffic engineering}

QoS-Label from IP is mapped to a MPLS-Label-field.


\section{IPv6}

Dualstack Issues: IPv6 wurde zuerst probiert, wenn aber keine Internet-Verbindung v6 existiert, wude nach 20-30s timeout v4 genutzt.

\subsection{Header}
Fixe Header-Länge von 40 Bytes
Keine Fragmentierung
Keine Checksumme, machen die anderen Layers schon.

\begin{description}
\item[Traffic Class] \hfill \\
	
\item[Flow Label] \hfill \\
	Ersatz für Type of Service; QoS Feld.
	
\item[Payload Length] \hfill \\

MTU: Wenn das Inhaltliche Paket zu grosse ist, gibt es eine ICMPv6 Fehlermeldung.
	
\item[Next Header] \hfill \\
	Ersatz für Protocol Field (Payload Type).
	Können neu Gestacked werden; Mehrere Subheaders and Data. (z.B. mehrere TCP pakete in einem IP paket).
	
	
	
\item[Hop Limit] \hfill \\
\end{description}

Extension Headers: ICMPv6, Encryption etc.

\subsection{Addresses}

\begin{description}
	\item[Link local] Intern
	\item[Unique Local] Firmenintern
	\item[Global] Global.
\end{description}

\subsubsection{Casts}

\begin{description}
\item[Unicast] One to on

\item[Multicast] One to many

\item[Anycast] One-to-one-of-many: Physisch nächster Server.
\end{description}

Keine Broadcast-Adressen.


\subsubsection{Aggregatable Global Unicast Addresses}

%TODO: Screenshot Folie Aggregatable Global UNicast Addresses

IANA has /16
RIPE has /23
Proivder has /32
Customer has /48

\subsubsection{EUI-64}

%TOOD: Screenshot Slide

\subsubsection{Unique-Local Addresses}

Starten mit: fc00::/7

\subsubsection{LinklLocal Addresses}

FE80::/10

(Eselsbrücke: Ethernet)

\subsubsection{Multicast Addresses}

On the local link

FF00::/8

Last 32 Bit werden für die Ethernet-Multicast-Addresse benutzt.

\paragraph{Solicited-Node Multicast Address}

Informationen zur geplanten Zielgruppe sind in der Adresse.

Wird für Duplicated address detection verwendet.


24-bit meiner Adresse wird an \textbf{ff02::00:1:FF:\textit{BITS}}



%TODO: Slide IPV6 Multicast Address overview


\subsection{ICMPv6}

\subsubsection{Address Resolution}

ARP is now part of ICMPv6 (type 135 136)

Router listen on Multicast Groups (FF02::1::FF00:1)

Hosts do MAC / IP resolution thereover

\subsubsection{Router Advertisment}

(Type 134)

Contains: Prefix and lease time, autoconfig flag (dhcp y|n)

\subsubsection{Duplicated Address Detection}

Try to ''ping'' address; if it got no reply, it trusts that the address does not exist.

\subsubsection{Redirects}

Advice Host to send package to a other host.

\subsection{DNS / DHCPv6}

There are now AAAA records; many DNS servers answer with v4 and v6 simultanously

Multicast: FF05::1:3

Usual resolution (like DHCPv4)

\subsection{Routing}

RIPng, OSPFv3, IS-IS for IPv6, MP-BGP4, EIGRP for IPv6

Router have two separate daemons running for routing.

\subsection{Transition v4 to v6}

\subsubsection{Dual-Stack}
Run both v4 and v6.

Sockets are informed, what protocol is required.

\subsubsection{6to4}

Some sort of NAT, but based on IPv4-Addresses in 192.168.xx range.

\subsubsection{NAT-PT}

deprecated!

Possible to NAT v4 and v6.

\section{Optical Networks}

Multimode: 50$\mu$m for step index; 62.5 $\mu$m for graded index
Bit rate-distance product: $>500$MHz-km

Single-mode: $9\mu$m;
Bit rate-distance product $>100$ THz-km


Eye-Diagram shows the difference between the 1 and 0. Same goes for the width.

\subsection{Light absorbion / semaring}

Absorption by the fiber material
Scattering (Streuung)of the light from the fiber

Not all waves travel at exactly the same ''speed'', there happens a smearing

Dispersion can be compensated a bit by dispersion shifting.


Multiple frequencies have a (small) influence on each other.


\subsection{FTTH}

\subsubsection{PON}

One fiber to district, splitting to different places; everything is sent to every receiver.

Upstream is divided into slots.

CATV is overlayed.

Building is cheaper, running-cost is higher.

\subsubsection{PtP}


Many fibers

\subsubsection{}


\section{Storage}

Es wird unterschieden zwischen Block-Level Storage (iSCSI, FibreChannel) und File level Storage (NFS, SMB etc.) 

NAS: über shared netzwerk

SAN: über dedicated storage network

\subsection{SCSI}

SAN Fibre Channel is based on SCSI (Standard for block-based access)

LUN: ''Virtual HDD''

\subsection{Fibre Channel}

%TODO: Slide With Port types here


Addressing: World Wide Name (WWN) 64 or 128 bit; every interface has a WWPN (port number).

All Endports regsiter at Switch on address FF FF FC

Similar system like DNS to query paths<

When sending a package, a Token is reserved out of a counter. Only when a acknowledge is received, the token is set free again.



Largest Domain Size is technically 239. Practically never more than 75 are connected out of performance reasons.



Virtual SANs: Combine multiple SAN's in a Virtual network. A VSAN may contain multiple zones. A zone is a limited access right.

\end{document}