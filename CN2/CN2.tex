% This is a default-selection of plugins that are used widely in this repo.

\documentclass[a4paper,10pt,fleqn]{article}
\usepackage[utf8]{inputenc}

% deutsche Trennmuster etc.
\usepackage[ngerman]{babel}
\usepackage[T1]{fontenc}

% mathematical simbols and fonts
\usepackage{mathtools} 
\usepackage{amssymb}
\usepackage{amsmath}
\usepackage{ntheorem}
\usepackage{polynom}
\usepackage{marvosym}
\usepackage{tabu}
\renewcommand*{\bmod}{\mathbin{\%}}
\everymath{\displaystyle}

\usepackage{multicol}
\usepackage{color}
\usepackage[usenames,dvipsnames]{xcolor}
\setlength{\columnsep}{1cm}
\setlength{\columnseprule}{0.25pt}
\def\columnseprulecolor{\color{gray}}
\usepackage{hyperref}

\usepackage[margin=1.5cm]{geometry}
\usepackage{graphicx}
\usepackage{pgfplots}
\pgfplotsset{compat=1.10}

%Code higlighting

\usepackage{minted}

% make lists more compact:
\newlength{\wideitemsep}
\setlength{\wideitemsep}{.5\itemsep}
\addtolength{\wideitemsep}{-5pt}
\let\olditem\item
\renewcommand{\item}{\setlength{\itemsep}{\wideitemsep}\olditem}
\renewcommand{\arraystretch}{1.25}

\title{Zusammenfassung CN2}
\author{Fabian Hauser}
 
\begin{document}
\maketitle

\section{Varia}
10 Seiten A4 Zusammenfassung

\section{Telefonie}

\begin{description}
\item[Data Plane] \hfill \\
	Forward IP-Packets. Z.B. STP, VLANs
\item[Control Plane] \hfill \\
	Routing Protocols, Overlay technologies (MPLS), z.B: Forwarding of Ethernet Frames
\end{description}


\subsection{G.711 / PCM}
 256 Audio-Steps using 8 Bits, Sample 8000/sec (nyquist frequency) = DS0, 64Kbps


\subsection{ISDN}
Braucht NT Wandler.

\begin{description}
\item[S-Bus] \hfill \\
	 Digitaler Bus im Haus; Data Plane.
\item[D-Kanal] \hfill \\
	Synchronisierungs-Kanal zur Dorfzentrale
\item[PBX] \hfill \\
	Die Dorfzentrale; Übersetzt Rufnummer-Anfragen (Wählton) zu weiteren Zentralen

\end{description}

\subsection{SS7}

Ähnlich BGP ist das SS7 Routing Protokoll die Verbindung zwischen allen Telefonzentralen.
	
\subsection{SIP}
Läuft über Router; muss von Telefonnummer eine IP-Addresse "extrahieren".
	

\end{document}