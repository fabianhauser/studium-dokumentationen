%% This is a default-selection of plugins that are used widely in this repo.

\documentclass[a4paper,10pt,fleqn]{article}
\usepackage[utf8]{inputenc}

% deutsche Trennmuster etc.
\usepackage[ngerman]{babel}
\usepackage[T1]{fontenc}

% mathematical simbols and fonts
\usepackage{mathtools} 
\usepackage{amssymb}
\usepackage{amsmath}
\usepackage{ntheorem}
\usepackage{polynom}
\usepackage{marvosym}
\usepackage{tabu}
\renewcommand*{\bmod}{\mathbin{\%}}
\everymath{\displaystyle}

\usepackage{multicol}
\usepackage{color}
\usepackage[usenames,dvipsnames]{xcolor}
\setlength{\columnsep}{1cm}
\setlength{\columnseprule}{0.25pt}
\def\columnseprulecolor{\color{gray}}
\usepackage{hyperref}

\usepackage[margin=1.5cm]{geometry}
\usepackage{graphicx}
\usepackage{pgfplots}
\pgfplotsset{compat=1.10}

%Code higlighting

\usepackage{minted}

% make lists more compact:
\newlength{\wideitemsep}
\setlength{\wideitemsep}{.5\itemsep}
\addtolength{\wideitemsep}{-5pt}
\let\olditem\item
\renewcommand{\item}{\setlength{\itemsep}{\wideitemsep}\olditem}
\renewcommand{\arraystretch}{1.25}
% This is a default-selection of plugins that are used widely in this repo.

\documentclass[a4paper,10pt,fleqn]{article}
\usepackage[utf8]{inputenc}

% deutsche Trennmuster etc.
\usepackage[ngerman]{babel}
\usepackage[T1]{fontenc}

% mathematical simbols and fonts
\usepackage{mathtools} 
\usepackage{amssymb}
\usepackage{amsmath}
\usepackage{ntheorem}
\usepackage{polynom}
\usepackage{marvosym}
\usepackage{tabu}
\renewcommand*{\bmod}{\mathbin{\%}}
\everymath{\displaystyle}

\usepackage{multicol}
\usepackage{color}
\usepackage[usenames,dvipsnames]{xcolor}
\setlength{\columnsep}{1cm}
\setlength{\columnseprule}{0.25pt}
\def\columnseprulecolor{\color{gray}}
\usepackage{hyperref}

\usepackage[margin=1.5cm]{geometry}
\usepackage{graphicx}
\usepackage{pgfplots}
\pgfplotsset{compat=1.10}

%Code higlighting

\usepackage{minted}

% make lists more compact:
\newlength{\wideitemsep}
\setlength{\wideitemsep}{.5\itemsep}
\addtolength{\wideitemsep}{-5pt}
\let\olditem\item
\renewcommand{\item}{\setlength{\itemsep}{\wideitemsep}\olditem}
\renewcommand{\arraystretch}{1.25}

\title{Zusammenfassung An2I}
\author{Fabian Hauser}
 
 %  -> \dt
 \newcommand{\dt}{\text{ }\mathrm{d}t}
 
\begin{document}
\maketitle

\section{Taylorreihen}
Jede ableitbare Funktion lässt sich durch Polynome approximieren.


\paragraph{Beispiel eines Polynoms 3. Ordnung}
\begin{align*}
	p(x_0) & = a(x_0-x_0)^3 + b(x_0-x_0)^2 + c(x_0-x_0) + 0 &= d = 0!d = 1d \\
	p'(x_0) &= 3a \cdot (x_0-x_0)^2 + 2b * (x_0-x_0) + c + 0 &= 1 \cdot c = 1!c = 1c \\
	p''(x_0) &= 2 \cdot 3a (x_0-x_0) + 2b + 0 + 0 &= 1 \cdot 2b = 2!b = 2b \\
	p^{(3)}(x_0) &= 2 \cdot 3a + 0 + 0 + 0 &= 1 \cdot 2 \cdot 3a = 3!a = 6a \\
	p^{(4)}(x_0) &= 0 + 0 + 0 + 0 &= 0
\end{align*}
was bedeutet:
\begin{align*}
	d &= \dfrac{p(x_0)}{0!} \\
	c &= \dfrac{p'(x_0)}{1!} \\
	b &= \dfrac{p''(x_0)}{2!} \\
	a &= \dfrac{p^{(3)}(x_0)}{3!}
\end{align*}

Wenn wir folgendes Polynom vervollständigen möchten:
\begin{align*}
	p(x) &= 2x^3 - 5x + 7 = [\frac{12}{3!}](x-1)^3 + [\frac{12}{2!}](x-1)^2 + [\frac{12}{1!}](x-1) + [\frac{4}{0!}] \\
	p'(x) &= 6x^2 - 5 => p'(1) = 6 * 1^2 -5 ) =1 \\
	p''(x) = 12x => p''(1) = 12 \\
	p^{(3)}(x) = 12 => p^{(3)}(1) = 12
\end{align*}
(es wird jeweils das Polynom an der Position 1 entwickelt)

\paragraph{Beispiel II}

Ausgangslage: Funktion $f(x)$, Entwicklungspunkt $x_0$, Ordnung $\mathbb{N}$ \hfill \\

Taylorpolynom der Ordnung $\mathbb{N}$ : 
%TODO: Cleanup next equation
\begin{align*}
P_N(x) &= f(x_0)/0! + f'(x_0)/1! * (x - x_0) = f''(x_x0) / 2! * x(-x_0)^2 + ... + f^(n)(x_0)/n! (x-x_0)^n \\
&= \sum^N_{k=0}{\frac{k^{(k)}(x_0)}{k!} (x-x_0)^k}
\end{align*}

Übrigens: Ein Taylorpolynom der 1. Ordnung ist die Linearisierung:
\begin{align*}
	f(x_0) + f'(x_0)(x-x_0)
\end{align*}

Feststellung: für $x \approx x_0$ gilt $f(x) \approx P_N(x)$, je grösser N, umso besser.


\paragraph{Beispiel III}

\begin{align*}
	f(x) &= e^x & \text{Entwicklungspunkt: } x_0=0 \\
	f'(x) &= e^x &= f(0) = 1\\
	f^{(n)}(x) &= e^x &= f^{(n)}(0) = 1
\end{align*}
was im Taylorplynom resultiert:
\begin{align*}
e^x \approx \frac{1}{0!} + \frac{1}{1!}(x-0)^1 + \frac{1}{2!}(x-0)^2 + \frac{1}{3!}(x-0)^3 + ...
\end{align*}

Eine Taylorreihe ist ein Taylerpolynom der $\infty$ Ordnung, und genau gleich wie die Ursprungsfunktion.

\emph{Konvergenzradius}: Distanz vom entwicklungspnukt, bis den das Taylorpolynom gilt. Im Fall von $e^x$ ist dieser $\infty$ gross.

\subsection{Fehlerabschätzung}

Gesucht: $f(x)$, Entwicklungspunkt $x_0$, Ordnung N des Taylorpolynoms:

\[
	P_N(x) = \sum^{N}_{k=0}{\frac{f^{(k)}(x_0)}{k!} (x-x_0)^k}
\]

Rechenfehler: $\left| f(x) - P_N(x) \right| \leq \frac{m}{N+1)} \left| x-x_0 \right|^{N+1}$ mit $m \geq \left| f^{(N+1)}(t) \right|$

$t$: Punkt zwischen $x$ und $x_0$, an dem $\left| f^{(N+1)}(t) \right|$ genau der Differenz entspricht.

Ziel: $f(x)$ im Intervall $[a,b]$ mit maximalem Fehler $\Delta$ zu bestimmen.
Suche Zahl $m \leq$ max $f^{(N+1)}(x)$ für $x \in [a,b]$. $a$ und $b$ ist der mögliche Wertebereich der Funktion $f(x)$
%TODO: Siehe Script, vervollständigen

\subsubsection{Abschätzung}



\paragraph{Beispiel}

Ziel: Sinus im Intervall $[0, \pi]$ mit Genauigkeit $\Delta = \frac{1}{100}$

\begin{align*}
	f(x) &= \sin(x) \\
	f'(x) &= \cos(x) \\
	f^{(2)} &= -\sin(x) \\
	f^{(3)} &= -\cos(x) \\
	f^{(4)} &= \sin(x) \\
\end{align*}

Beim Sinus gilt:
\[
	\left| f^{(N+1)}(x) \right| \leq m = 1
\]

\[
	\left| sin(x) - P_N(x) \right| \leq \frac{1}{(N+1)^1} \left(\frac{\pi-0}{2}\right)^{N+1} < \frac{1}{100}
\]

\subsection{Linearisieren / Taylorpolynom 1. Ordnung)}
Gesucht: $f(x), x_0$: Ziel Linearisierung um $x_0$

Antwort: $f(x) \approx f(x_0) + f'(x_0)(x-x_0) = P_1(x)$

In welchem Intervall ist der Rechenfehler $< \Delta$?

\[
	\left|f(x) - p_1(x) \right| \leq \frac{m}{2!}(?)^2 < \Delta \\
\]

\[
	|?| < \sqrt{\frac{2 \Delta}{m}}
\]

Antwort für $x \in \left[ x_0 - \sqrt{\frac{2 \Delta}{m}}; x_0 + \sqrt{\frac{2 \Delta}{m}} \right] = I_m$

Was ist also $m$? Globales Maximum von $|f''(x)|$ im Intervall $I_m$
Problem: es gibt eine Rückkopplung. \\
Angenommen werden kann ein Wert von ca. $|f''(\text{ca. } 1.5 \cdot x_0)|$


\subsubsection{Beispiel}
	EP: $x_0 = 1$
	
\begin{align*}
	f(x) &= sin(x^2) &\rightarrow f(1) = sin(1) \\
	f'(x) &= cos(x^2) \cdot 2x &\rightarrow f'(1) = 2 cos(1) \\
	f''(x) &= -sin(x^2) \cdot (2x)^2 + cos(x^2) \cdot 2 &\rightarrow f''(1) = 2 cos(1) - 4 sin(1)
\end{align*}

\[
	f(x) \approx sin(1) + 2 cos(1) (x-1) = P_1(x) 
\]

Da $|f(x) - p_1(x)| < \Delta$, gilt das Intervall für $x \in \left[1-\sqrt{\frac{2 \Delta}{m}}; 1+\sqrt{\frac{2 \Delta}{m}} \right] = I$.

\[
	f''(1) \approx 2,29
\]
Hypothese: $m=3 \approx 1.5 \cdot 2.29$

\[
	I = \left[1-\sqrt{\frac{0.002 \Delta}{3}}; 1+\sqrt{\frac{0.002 \Delta}{3}} \right]
\]

Test: $|f''(x)| \leq 3$ für all $x \in I$ ?

\subsection{Konvergenzradius}

Bereich um den Entwicklungspunkt, in dem die Taylorreihe den richtigen Wert der Ausgangsfunktion liefert.

Üblicherweise ist der Konvergenzradius einer Taylorreihe gleich dem Abstand vom Entwicklungspunkt zur ersten Definitionslücke (in den Reelen oder Komplexen Zahlen).

\section{Grenzwerte}

\subsection{Eigentliche Endwerte im Unendlichen}
Grenzwert ist gleich Null: $f(x) \xrightarrow{x \rightarrow \infty} 0$

Limes: $\lim_{x \rightarrow \infty}{f(x)} = 0$

Uneigentliche Grenzwerte sind $\infty$, was aber streng genommen keine Zahl ist.

\paragraph{Beispiele:}

\begin{align*}
	\lim_{x \to \infty}{arctan(x)} &= \frac{\pi}{2} \\
	\lim_{x \to -\infty}{arctan(x)} &= -\frac{\pi}{2}
\end{align*}

\subsubsection{Systematische Analyse}

der Funktion $f(x): \frac{\sin{(x)}}{x}$. Behauptung: $ \lim_{x \to \infty}{f(x)} = 0$

Streifen um den angenommenen Grenzwert 0. Der "Radius" wird als $\epsilon$ bezeichnet.

\paragraph{Grenzzahl $X(\epsilon)$,} ab dem der Funktionswert innerhalb dieser Zahl liegt. Wenn es keine Rolle spielt, wie gross $\epsilon$ ist, um ein Resultat zu liefern, stimmt der angenommene Grenzwert.

\paragraph{Beweis (Definition Grenzwert):} Für jede noch so kleine Zahl $\epsilon > 0$ liegt die Funktion ab einer Stelle $X$ (bzw. $X(\epsilon)$) vollständig in einem Streifen der vom Grenzwert $0$ um jeweils ein $\epsilon$ nach oben und unten weggeht.
\[
\text{D.h.:  alle } x > X(\epsilon) \text{ gilt: } |f(x)-0| < \epsilon
\]
Auf jeden Fall gilt, $X = \frac{1}{\epsilon}$

Trick zur Grenzwertanalyse bei z.B. $x^2$: Umkehren zu $\frac{1}{x^2}$, was qualitativ gleich aussieht wie $\frac{1}{x}$

\subsection{Eigentliche Grenzwerte im Endlichen}

\paragraph{Beispiel:} von $\lim\limits_{x \to 2}{\frac{x^2 + x -6}{x-2}}$.

Vorgehen: Stetige Fortsetzung suchen, Zahlenwert bei Definitionslücke  einsetzen.


Wenn die Funktion an der gesuchten Stelle definiert ist, wird dieser wert daraus entfernt. So wird zum Beispiel aus $\lim\limits_{x \to 0}{sig^2(x)} = 1$ wird $\lim\limits_{x \to 0}{\text{sig}^2_{|\mathbb{R} \setminus \{0\}}(x)}$

\subsection{Einseitige Grenzwerte}

	Beachtet nur ein Teilbereich der Funktion; Beim rechtsseitige Grenzwert werden die kleineren Zahlen ignoriert:
\[
	\lim\limits_{x \to 0+}{\text{sig}(x)} = \lim\limits{x \to 0} \text{sig}|_{\mathbb{R} \setminus [0;\infty)}(x) = 1
\]
bzw. die grösseren Zahlen ignoriert:
\[
	\lim\limits_{x \to 0-}{\text{sig}(x)}
\]

\subsubsection{Rechentechniken}

\paragraph{Subfunktionen}

\paragraph{Unstetige Stellen}

\[
	\lim\limits_{x \to 0}(x \cdot \text{sig}(x)) = 0
\]

$x$ geht gegen Null; sig$(x)$ \emph{existiert nicht} an der Stelle 0. Ein möglicher Weg:
\[
	\lim\limits_{x \to 0+}(x \cdot \text{sig}(x)) = 0
\]

$x$ ist nach wie vor 0; Beim Signum können wir 1 annehmen (da positive Zahl). Beim umgekehrten Weg:
\[
	\lim\limits_{x \to 0-}(x \cdot \text{sig}(x)) = 0
\]

können wir $x$ als 0 annehmen; Beim Signum können wir $-1$ annehmen (da negative Zahl).

Der beidseitige Grenzwert ist gleich; daraus können wir schliessen, dass der Grenzwert auch $0$ ist.

In diesem Fall wäre auch die Analyse der eigentlichen Funktion; in diesem Fall steht nur ein komplizierter Betrag.

\paragraph{Gegen unendlich}

\[
	\lim\limits{x \to \infty} \sin(\frac{\pi x^2 + 3x - 4}{4x^2-4})
\]
Zähler und Nenner bewegen sich gegen unendlich. Lösungsansatz: Schnellst anwachstende Komponente ausklammern:

\[
	\lim\limits_{x \to infty} \sin(\frac{x^2(\pi + \frac{3}{x} - \frac{4}{x^2})}{x^2(4 - \frac{4}{x^2})}) =
	\lim\limits_{x \to infty} \sin(\frac{\pi + \frac{3}{x} - \frac{4}{x^2}}{4 - \frac{4}{x^2}}) = sin(\frac{\pi}{4}) = \frac{\sqrt{2}}{2}
\]


\subsection{Uneigentliche Grenzwerte}

$\infty$ und $-\infty$ zerstören die üblichen Zahlenkörper; deshalb gelten damit die normalen Rechengesetze nicht mehr. Deshalb: Unendlich ist eher Sammelbegriff für "sehr viel", aber es ist nicht genau klar, "wie viel".


\subsubsection{Beispiele}
Sehr grosse Zahl in Exponentialfunktion einsetzten, ergibt eine sehr grosse Zahl.
\[
	\lim\limits_{x \to \infty} e^x = \infty
\]

Konstante $C$, über welcher die Funktion am Ort $X(C)$ ist.

In diesem Fall: Für alle $x > \ln(x)$ gilt $x^x > e^{\ln(C)} = C$


\subsubsection{$\infty$}

Unklar: $\infty - \infty$
B/L: $\infty \cdot (\pm 0)$; $\frac{\infty}{\infty}$; $\frac{0\pm}{0\pm}$

\[
	\lim\limits_{ x \to \infty} \frac{x^2-3x -7}{\sqrt{11x^4+3x+9}} \left(\approx \lim\limits_{ x \to \infty} \frac{(\infty - \infty) - 7}{\infty}\right) =
	 \lim\limits_{ x \to \infty} \frac{x^2 (1 - \frac{3}{x} - \frac{7}{x^2}}{\sqrt{11x^4 + 3x + 9}} \left(\approx \lim\limits_{ x \to \infty} \frac{\infty}{\infty}\right) =
	 \lim\limits_{x \to \infty} %TODO: Multiplikation mit 1/x^2
	 = \frac{1}{\sqrt{11}}
\]

\subsection{Regel von Bernoulli / Lobital}

\[
	\lim\limits_{x \to q} \frac{f(x)}{g(x)} =^{\text{Typ } \frac{0}{0}} \lim\limits_{x \to q} \frac{f'(x)}{g'(x)} %TODO: =-text anpassen
\]


\subsection{Ableitung mit dem limes}

\[
	f'(x) = \frac{df}{dx}|_{x=x_0} = \lim\limits_{x \to x_0} \frac{f(x) - f(x_0)}{x - x_0} =^\frac{0}{0} = \lim\limits_{x \to x_0} \frac{\frac{d}{dx}(f(x) - f(x_0))}{\frac{d}{dx}(x - x_0} = \lim\limits_{x \to x_0} \frac{f'(x) -0}{1 - 0} = f'(x_0)
\]

\section{Integral}

\subsection{Stammfunktionen}

Umkehrung der Ableitung.

Gegeben: Funktion $f(x)$

Gesucht: Funktion $F(x)$ (da Stammfunktion, stetige Funktion)

Eigenschaft: $F'(x) = f(x)$ für all $x$ (aus dem Definitionsbereich von $x$ in $f(x)$)

Dann heisst $F$ Stammfunktion von $f$.

\paragraph{Beispiel} \hfill

$f(x) = \sin(x)$

$F(x) = -\cos(x) + c$

Vergewissern: $F'(x) = \frac{\mathrm{d}}{\mathrm{d}x}(\cos(x) + c) = -\sin(x) = f(x)$

Mögliches vorgehen: «dirty» Ableitung rückgängig machen, prüfen, ob sie beim Ableitung richtig rauskommen, nachkorrigieren.

\subsection{Unbestimmtes Integral}

Das Unbestimmte Integral dient immer der Suche nach der Stammfunktion.
Sei $f(x)$ eine Funktion und $F(x)$ eine Stammfunktion von $f$.

 \[
 \int \underbracket{f(x)}_{\mathclap{\text{Integrand}}}\, \overbracket{\mathrm{d}x}^{\mathclap{\text{Integrationsvariable}}} = F(x) + \underbracket{\text{ const}}_{\mathclap{\text{Integrationskonstante}}}
 \]
 
 ist $\Leftrightarrow$ zu:
 
 \[
	 f(x) = \frac{\mathrm{d}}{\mathrm{d}x}(F(x))
 \]


$+\text{const}$ bedeutet: Die Zahlen $a + \text{const} = b + \text{const}$ sind $a$ und $b$ in der gleichen Restklasse (d.h. es gibt einen Modulo-Konstante). Durch das $+\text{const}$ wird aus der Funktion eine Menge.

\subsubsection{Hauptsatz}

(Üblicherweise wird das Integral via dem $\lim$es und diesen beiden Hauptsätzen definiert.)

\begin{align*}
	f(x) = \frac{\mathrm{d}}{\mathrm{d}x} (F(x) + const) &= \frac{\mathrm{d}}{\mathrm{d}x} \int f(x)\mathrm{d}x \\
	F(x) + const &= \int F'(x) \mathrm{d}x
\end{align*}

\subsubsection{Wichtige Integralregeln}

\paragraph{$1/x$}
\[
	\int \frac{1}{x}\mathrm{d}x = \ln(|x|) + \mathrm{const}
\]

\paragraph{$f'(x) / f(x)$}
\[
	\int \frac{f'(x)}{f(x)}\mathrm{d}x = \ln(\left| f(x) \right|) + const
\]

\paragraph{$\ln(x)$}
\[
	\int \ln(u) \mathrm{ d}u = u \cdot \ln(u) - u + \text{const}
\]

\paragraph{$f'(ax+b)$}
\[
	\int f'(as + b) \mathrm{d}s = \frac{1}{a}f(as + b) + \mathrm{const}
\]

Siehe auch Seite 54 im Skript oben, sowie häufige Pattern auf Seite 57.

\paragraph{Die Partielle Integration}
hilft, wenn eines der beiden folgenden Integrale einfacher gelöst werden kann:

\[
	\int{f(x) g(x)} \mathrm{d}x = F(x) g(x) + \mathrm{const} - \int{F(x) g'(x) }\mathrm{d}x
\]

Wird oft verwendet, wenn $g(x)$ ein Polynom ist, da damit der Polynomgrad gesenkt werden kann.

\subsection{Bestimmtes Integral}

Ist eine Zahl, welche auch für eine andere Stammfunktion gleich bleib. Der Integrand muss zwingend im Intervall $[x_1; x_2]$ definiert sein, sonst existiert dafür (vermutlich) kein bestimmtes Integral.
\[
\int\limits^{x_1}_{x_2} = F(x_1) - F(x_2) = (F(x_1) + c) - (F(x_2) + c) = \hat{F}(x_1) - \hat{F}(x_2)
\]
Was in der Praxis folgendermassen aufgeschrieben wird:
\[
	\int\limits^2_1 x \mathrm{ d}x = \left[ \frac{1}{2} x^2 \right]^2_{x=1} = \frac{1}{2} 2^2 - \frac{1}{2} 1^2
\]

Das tauschen der Integrationsgrenzen, ändert das Vorzeichen des Terms. Wenn $x_1$ und $x_2$ gleich sind, ist die Integration gleich $0$. Es ist möglich, Integrale von Teilbereichen zu berechnen und die Ergebnisse aufsummieren.

\subsubsection{Definition über den Grenzwert}

\[
 \int\limits^1_0{f(t)\mathrm{d}t} = \lim_{n \to \infty} \sum^n_{i=1}{f(\frac{i}{n}) \cdot \frac{1}{n}}
\]

\subsubsection{Definition über den Flächeninhalt}

Fläche unter einem Graphen vom zwischen $[a;b] \in x$-Achse berechnen.

Das Vorgehen der Näherung: Unterteilung der Fläche in kleine Rechtecke $A_1$ bis $A_n$, was zu einer ähnlichen Funktion $f^\ast(x)$ führt.


\[
A_K = \text{Länge} \cdot \text{Höhe} = \frac{b-a}{n} \cdot f\left(a + k \cdot \frac{b-a}{n}\right)
\]

\[
A^\ast = \sum^n_{k=1}{A_k} = \frac{b-a}{n} \cdot \sum^n_{k=1}{ f\left(a + k \cdot \frac{b-a}{n}\right)}
\]

Natürlich wird immer genauer, je kleiner die Rechtecke $A_k$ bzw. grösser das $n$ gewählt wird; am besten unendlich klein.

\[
A = \int\limits^b_a{f(x) \mathrm{d}x} = \lim_{n \to \infty} \left( \frac{b-a}{n} \cdot \sum^n_{k=1}{ f\left(a + k \cdot \frac{b-a}{n}\right)} \right)
\]

Dies funktioniert auch bei einigen unstetigen Funktionen, es sind allerdings auch Funktionen möglich, welche nicht funktionieren.

Wichtig: Negative Flächen werden von positiven Flächen abgezogen.

\subsubsection{Integralfunktion}

\begin{align*}
	\int\limits^{x}_{1} \frac{1}{t} \mathrm{d}t &= \left[ \ln(|t|) \right]^x_1 = \ln(x) - \ln(1) & \text{für } x > 0
\end{align*}

\subsubsection{Geschwindigkeiten}

\begin{description}
	\item[$v(t)$] Momentangeschwindigkeit 
	\[
		\bar{v} = \frac{s}{T_2- T_1} = \frac{\int\limits^{T_2}_{T_1}{v(t)\mathrm{d}t}}{T_2 - T_1}
	\]
	\item[$\bar{v}$] Durchschnittsgeschwindigkeit
\end{description}

Gesucht: Mittlerer Funktionswert im Intervall $[a;b]$, $f(t)$
\[
	\bar{f} = \frac{1}{b-a} \underbrace{\int\limits^b_a{f(x)\mathrm{d}x}}_I
\]

Rechteck mit gleicher Fläche $I = (b-a) \cdot \bar{f}$

\section{Fourierreihe}

Fourierreihen approximieren eine Funktion im Gesamtbild, nicht um einen einzelnen Punkt wie bei der Taylorriehe.

\subsubsection{Vergleich Taylor- und Fourrierreihe}

\begin{tabu} to \linewidth {l | X X }
& \textbf{Taylor} & \textbf{Fourier} \hfill \\
\hline

\textbf{Input} & Funktion (oft ableitbar), Entwicklungspunkt $x_0$ & Periodische Funktionen \\
\hline

\textbf{Approxim.} & Polynome $a_0  + a_1 (x-x_0) + a_2(x-x_0)^2 + ...$ & Sin- und cos-Terme $a_0+a_1 \cos(x)+b_1 \sin(X) + a_2 \cos(2x) + b_2 \sin(2x) + ...$ \\
\hline

\textbf{Bausteine} &$(x-x_0)^k; k \in \mathbb{N}$ &$\sin(kx), \cos(kx), k \in \mathbb{N}$ \\
\hline

\textbf{Approximations-verhalten} & vom Entwicklungspunkt (''innen'') weg (nach ''aussen''); wird immer besser & global (von groben zu feinen Strukturen). keine exakte Berechnung einzelner Punkte \\
\hline

\textbf{Berechnung der Koef.} & durch Ableiten am Entwicklungspunkt. & versuch, die Fehlerfläche zu minimieren \\ % Lüge! Ziel: Quadratische Fsläch minimieren.
\lasthline
\end{tabu}

\paragraph{Vorgehen:}
Integrieren von $|s_\text{ourier-approx-funktion}(x)-f_\text{approx}(x)|$, wobei $f_\text{approx}$ so gewählt wird, dass die Fläche möglichst klein wird.% Lüge! siehe Tabelle

Um u.a. den Betrag weg zu kriegen, wird 
\[
	\int\limits^1_{-1}(s(x)-f(x))^2 \mathrm{d}x
\]


\subsubsection{Beispiel annäherung Rechtechsignal}

%TODO: Bild aus Script Seite 11

Der Sinus und die Rechtechfunktion haben eine kompatible Periode, was das Berechnen vereinfacht.

Ziel: Finden einer $f(t) = A \cdot \sin(\underbracket{\omega}_{\mathclap{\text{Grundkreisfequenz}}} t)$, so das sie $s(t)$ möglichst gut approximieren.

\begin{enumerate}
	\item Bestimme $\omega$ so, dass $f(t)$ diesselbe primitive Periode wie $s(t)$ hat.
	\item Bestimme $A$ so, dass die quadratische Fehlerfläche möglichst klein wird:
		\[
			\int\limits^1_{-1}(s(t) - f(t))^2 \mathrm{d}t
		\]
		Ziel: Die mittlere Abweichung von $s(t)-f(t)= $ Mittelwert des Integrals $= \frac{\text{Integral}}{2}$ soll möglichst klein werden 
\end{enumerate}

\paragraph{Umestzung}

\begin{enumerate}
	\item	$f(t) = A \cdot \sin(\pi t)$ daraus folgt:
	\[
		Q = \int^1_{-1}\left(s\left(t\right) - A \cdot \sin\left(\pi t\right)\right)^2 \mathrm{d}t
	\]
	\item	Bestimme $Q$ in Abhängigkeit von $A$:
	\[
		Q = \int\limits{1}{-1} s^2(t) \mathrm{d}t -2A \int\limits{1}{-1} \cdot s(t) \sin(\pi t) \mathrm{d}t + A^2 \int\limits{1}{-1} \sin^2(\pi t) \mathrm{d}t = \text{Zahl} - 2A \cdot \text{Zahl} + A^2 \cdot 1 
	\]
	\[
		Q = 2 - 2A\left( \int\limits^0_{-1} -1 \sin(\pi t) \mathrm{d}t + \int\limits^1_0 1 \sin(\pi t) \mathrm{d}t\right) + A^2 = 2 -2A \frac{1 + 1 - (-1 -1)}{\pi} + A^2 = A^2 - \frac{8}{\pi} A + 2
	\]
	\item Minimum von Q finden. 
	\[
		\frac{\mathrm{d}Q}{\mathrm{d}A} = 2A - \frac{8}\pi = 0 \Leftrightarrow A = \frac{4}\pi
	\]
\end{enumerate}
\----
%TODO: Nächste Liste noch nötig
\begin{enumerate}
	\item	$a_0$ Fourierriehe der Ordnung 0: $\int\limits^0_{-1}(-1-a_0)^2 \mathrm{d}t + \int\limits^1_0(1-a_0)^2 \mathrm{d}t = (-1-a_0)^2 + (1-a_0)^2$
	
	
\end{enumerate}


\subsubsection{Periodizität}

Fourierreihen können nur mit periodischen Funktionen verendet werden.

\begin{description}
	\item[Primitive Periode:] Nächste Wiederholung bei $2\pi$ Abstand
	\item[Periode:] Wiederholung bei $2x\pi$ Abstand
\end{description}

\subsection{Herleitung Fuourier-Reihe}

Gegeben ist Periode $T$. ($f(t+T) = f(t)$ für alle $t \in \mathrm{R}$)

Welche Funktion vom Typ $f(t) = a \cdot \cos(\omega t)$ habe die Periode $T$?

\[
	a \cdot \cos(\omega t + \omega T) = a \cdot \cos(wt)
\]
für alle $t \in \mathbb{R}$. Wegschaffen a (Annahme: $a \neq 0$) und $\cos$:
\[
	(\omega t + \omega T = \pm \omega t + 2k\pi \text{ für alle } t \in \mathbb{R}), k \in \mathbb{Z}
\]
wenn nur die positive Lösung betrachtet wird (negative Gleichung produziert keine Ergebnisse):
\[
	\omega T =  k \pi, k \in \mathbb{Z}
\]

Daraus schlissen wir: $\omega = \frac{2 \pi}{T} \cdot k, k \in \mathbb{Z}$. Der Term $\frac{2 \pi}{T}$ wird als $\omega_1$ bezeichnet.

\subsubsection{Zentrale Integrale}

\paragraph{1}
\[
	\int\limits^T_0 \sin(k \omega_1 t) \cdot \sin(l \omega_1 t) \mathrm{d}t = \int\limits^T_0 \cos(k \omega_1 t) \cdot \cos(l \omega_1 t) \mathrm{d}t = 
	\begin{cases}
	0   & \text{für } k \neq l \\
	T/2 & \text{für } k = l
	\end{cases}
\]

\paragraph{2}
\[
	\int\limits^T_0 \sin(k \omega_1 t) \cdot \cos(l \omega_1 t) \mathrm{d}t = 0 \text{ für alle } k,l
\]

gilt für $\omega_1 = \frac{2\pi}{T}, k,l \in \mathbb{N} \setminus{\{0\}}$

\subsubsection{Flächenminimierung}


Gesucht: Signal $s$ mit Periode $T$, $\omega_1$

Ziel: $f(t) = a \cdot \cos(\omega_1 t) + b \sin(\omega_1 t)$

Quadratische Fehlerfläche:
\[
	\int\limits^T_0 \left(s(t) - f(t)\right)^2 \mathrm{d}t = \int\limits^T_0\left(s(t) - a \cos(\omega_1 t) - b \sin(\omega_1 t\right)^2 \mathrm{d}t
\]

\[
	= \underbracket{\int\limits^T_0 s^2(t) \mathrm{d}t}_\text{Irrelevant (da Konstante)} + a^2 \underbracket{\int\limits^T_0 \cos^2(\omega_1 t) \mathrm{d}t}_{=T/2} + \underbracket{\int\limits^T_0 \sin^2(\omega_1t)\mathrm{d}t}_{=T/2} \]\[
	- 2a \int\limits^T_0 s(t) \cos(\omega_1 t) \mathrm{d}t - 2b \int\limits^T_0 s(t) \sin(\omega_1 t) \mathrm{d}t + 2ab \underbracket{\int\limits^T_0 \cos(\omega_1 t) \sin(\omega_1t) \mathrm{d}t}_{=0}
\]

Minimum von $F$ finden mittels $\frac{dF}{da}=0$, $\frac{dF}{db} = 0$, hier mit $a$:

\[
	\frac{dF}{da} = T \cdot a - 2 \int\limits^T_0 s(t) \cos(\omega_1 t) \mathrm{d}t = 0
\]


\subsubsection{Konstruieren eine Fourierreihe}

Gemäss Bild Fourier-Skript Seite 20

\paragraph{$\omega_1$} \hfill \\

Periode $T=2 \Rightarrow \omega_1 = \frac{2 \pi}{T} = \pi$
\[
s(t) = \begin{cases}
	t & \text{ für } t \in [0,1) \\
	0 & \text{ für } t \in [1,2)
\end{cases}
\]
Bei Sprüngen liefert die Fourierriehe den Mittelwert der beiden Randstellen.

\paragraph{$a_0$}

\[
	a_0 = \frac{1}{T} \int\limits^T_0 s(t) \mathrm{d}t = \frac{1}{2} \int\limits^2_0 s(t) \mathrm{d}t \underbracket{=}_{\mathclap{\text{Kästchenzählen}}} \frac{1}{2} \cdot \frac{1}{2} = \frac{1}{4}
\]

\paragraph{$a_1$}

\[
	a_1 = \frac{2}{T} \int\limits^T_0 s(t) \cos(1 \omega_1 t) \mathrm{d}t
	 = \int\limits^2_0 s(t) \cos(\pi t) \dt
	 = \int\limits^1_0 \underbracket{s(t)}_{=t} \cos(\pi t) \dt + \int\limits^2_1 \underbracket{s(t)}_{=0} \cos(\pi t) \dt
	 = \int\limits^1_0 t \cos (\pi t) \dt
\]
Dieses letzte Integral müssen wir Ausrechnen, es lässt sich nicht weiter vereinfachen. Üblicherweise partielle Integration.
\begin{align}
	a_1 &= \int\limits^1_0 t \cos(\pi t) \dt \\
		&= \left[ 1 \cdot \frac{1}{\pi} \sin(\pi t) \right]^1_{t=0} - \int\limits^1_0 1 \cdot \frac{1}{\pi} \sin( \pi t) \dt \\
		&= \underbracket{(1 \cdot \frac{1}{\pi} \underbracket{\sin(\pi 1)}_{=0}) - 0}_{=0} - \frac{1}{\pi} \int\limits^1_0 \sin( \pi t) \dt \\
		&= \left[ - \frac{1}{\pi} \cos(\pi t) \right]^1_{t=0}  = \frac{1}{\pi^2} \left( \cos(\pi) - \cos(0)\right) = - \frac{2}{\pi^2}
\end{align}

\paragraph{$a_k$}

Die Berechnung von $a_1$ kann genau so für $a_k$ angewendet werden, nur dass im $\cos$ jeweils ein $k \pi$ statt ein $1 \pi$ steht.

In diesem Fall funktioniert das, weil der $\sin(k\pi)$ für alle $k \in \mathbb{Z}$ gleich $0$ ist. Ausnahme: Endergebnis $- \frac{2}{\pi^2}$ ist nicht möglich.

\[
	a_k = \frac{\cos(k \pi) -1}{k^2 \pi^2} = \begin{cases}
		0 & \text{ für } k \text{ gerade} \\
		- \frac{2}{k^2 \pi^2} & \text{ für } k \text{ ungerade}
	\end{cases}
\]


\paragraph{$b_k$} Für $b_k$ geht alles gleich wie beim $a_k$, bis auf den Schritt der partiellen Integral, halt mit $\sin$ statt $\cos$.

\begin{align*}
	b_k &= \frac{2}{2} \int\limits^2_0 s(2) \sin(k \pi) \dt 
		= \int\limits^1_0 \underbracket{s(t)}_{=t} \sin(k \pi t) \dt + \underbracket{\int\limits^2_1 s(t) \sin(k \pi t) \dt}_{=0} = \int\limits^1_0 t \sin(k \pi t) \dt \\
		&= \int\limits^1_0{t \sin(k \pi t)\dt} = \left[ t \cdot \left(-\frac{1}{k \pi}\right) \cos(k \pi t) \right]^1_{t=0} - \int\limits^1_0{ - \frac{1}{k \pi} \cos(k \pi t)\dt} \\
		&= -\frac{1}{k \pi} - 0 + \underbracket{\frac{1}{k \pi} \left[ \frac{1}{k \pi} \sin(k \pi t) \right]^1_{t=0}}_{\text{=0}} \\
		&= \frac{\cos(k \pi)}{k \pi} = \begin{cases}
			-\frac{1}{k \pi} & \text{ für } k \text{ gerade} \\
			\frac{1}{k \pi} & \text{ für } k \text{ ungerade}
		\end{cases}
\end{align*}

Damit lässt sich die Fourierreihe leicht berechnen:

\begin{align*}
f_n(t) &= a_0 + \sum^n_{k=1}{a_k \cos(k w t) + b_k \sin(k w t)} \\
	   &= \frac{1}{4} + \sum^n_{k=1}{ \frac{\cos(k \pi) -1}{k^2 \pi^2} \cos(k \pi t) - \frac{\cos(k \pi)}{k \pi} \sin(k \pi t)} \\
	   &= \underbracket{\frac{1}{4}}_{\text{0.}}
	    \underbracket{-\frac{2}{1^2 \pi^2} \cos(\pi t) + \frac{1}{1 \cdot \pi} \sin(\pi t)}_{\text{1. Ordnung}}
	    \underbracket{-\frac{1}{2 \pi} \sin(2 \pi t) -\frac{2}{3^2 \pi^2}}_{\text{2. Ordnung}}
	    \underbracket{-\cos(3 \pi t) + \frac{1}{3 \pi} \sin(3 \pi t)}_{\text{3. Ordnung}}
\end{align*} %TODO: w mit \omega ersetzen

\subsection{Satz von Dirichlet}

An unstetigen Stellen liefert die Fourierreihe den Durchschnitt der beiden Grenzwerten, an gewöhnlichen, stetigen Stellen stimmt die Fourierreihe genau.

\subsection{Amplituden-Phasen-Form}

Baut auf das zusammenführen des $\sin$ und $\cos$ Terms mithilfe der Additions-Theoreme. Die Phasen-Verschiebung wird in die Terme aufgenommen.

\begin{align}
	\cos(t - \varphi) &= \cos(t) \cos(\varphi) + \sin( t) \sin(\varphi) \\
	\cos(k \omega_1 t - \varphi_k) &= \cos(k \omega_1 t) \cos(\varphi_k) + \sin(k \omega_1 t) \sin(\varphi_k) \\
	A_K \cos(k \omega_1 t - \varphi_k) &= A_k \cos(k \omega_1 t) \cos(\varphi_k) + A_k \sin(k \omega_1 t) \sin(\varphi_k) \\
	A_0 + \sum^\infty_{k=1} A_K \cos(k \omega_1 t - \varphi_k) &= A_0 + \sum^\infty_{k=1} A_k \cos(k \omega_1 t) \cos(\varphi_k) + \sum^\infty_{k=1} A_k \sin(k \omega_1 t) \sin(\varphi_k) \\
	&= \underbracket{A_0}_{a_0} + \sum^\infty_{k=1} \underbracket{A_k \cos(\varphi_k)}_{a_k} \cos(k \omega_1 t) + \sum^\infty_{k=1} \underbracket{A_k \sin(\varphi_k)}_{b_k} \sin(k \omega_1 t)
\end{align}

Die letzte Gleichung Repräsentiert Fourierreihen; Rechts die Sinus-Cosinus-Form, Links die Amplituden-Phasen-Form ($A_k$ Amplitude, $\varphi_k$ die Phase).

\subsubsection{Umrechnung}

\paragraph{APF in SCF}
\begin{align*}
a_0 &= A_0 \\
a_k &= A_k \cos(\varphi_k) \\
b_k &= A_k \sin(\varphi_k)
\end{align*}

\paragraph{Sinus-Cosinus-Form in Amplituden-Phasen-Form}

\begin{align*}
A_0 &= a_0 \\
A_k &= \sqrt{a_k^2 + b_k^2} \\
\tan(\varphi_k) &= \frac{b_k}{a_K} \text{ für } a_k \neq 0 \\
\varphi_k &= \arctan\left(\frac{b_k}{a_k}\right) + \underbrace{m \pi}_{\mathrlap{\text{Was ist das richtige } m \text{?}}} \\
\varphi_k &= \underbrace{\begin{cases}
\arctan\left( \frac{b_k}{a_k} \right) & \text{ für } a_k > 0 \\
\pi + \arctan\left( \frac{b_k}{a_k} \right) & \text{ für } a_k < 0 \\
\frac{\pi}{2} & \text{ für } a_k = 0, b_k > 0 \\
-\frac{\pi}{2} & \text{ für } a_k = 0, b_k < 0 \\
\text{irrelevant} & \text{ für } a_k = b_k = 0 \\
\end{cases}}_{\text{Vereinfacht (Perioden ignoriert)}}
\end{align*}

Kein Standardverfahren für die nicht-lineare Gleichung... Vereinfachung: Analogie zu Koordinatensystemen. $\varphi_k$ als Winkel zur X-Achse, $A_k$ als Vektor, Polare Koordinaten. Der Punkt auf der y-Achse ist $b_k$, auf der x-Achse $a_k$.

\begin{tikzpicture}[scale=2,cap=round,>=latex]
	% draw the coordinates
	\draw[->] (-1.05cm,0cm) -- (1.1cm,0cm) node[right,fill=white] {$x$};
	\draw[->] (0cm,-1.05cm) -- (0cm,1.1cm) node[above,fill=white] {$y$};
	
	% draw the unit circle
	\draw[gray] (0cm,0cm) circle(1cm);
	
	% draw line
	\draw[very thick] (0cm,0cm) -- (35:1cm); % angle line
	\filldraw (35:1cm) circle(0.8pt); % point
	\draw (28:1.5cm) node {$A_K =(a_k; b_k)$}; % point label
	\draw (35/2-4:0.3cm) node {$\varphi_k$}; % angle description
	\draw (0.5cm,0cm) arc (0:35:0.5cm); %angle circle
	
	%x-y-anchor
	\draw[gray, dashed] ({cos(35)},0cm) -- (35:1cm);
	\draw ({cos(35)},-0.1cm) node {$b_k$};
	
	\draw[gray, dashed] (0cm,{sin(35)}) -- (35:1cm);
	\draw (-0.1cm,{sin(35)}) node {$a_k$};
	
\end{tikzpicture}


\paragraph{Berechnung:}

\begin{align*}
	A_0 &= \frac{1}{4} \\
	A_k &= +\sqrt{a_k^2 + b_k^2} = \begin{cases}
		\frac{1}{k \pi} & k \text{ gerade} \\
		\sqrt{\left(\frac{2}{k^2\pi^2}\right)^2 + \left(\frac{1}{k \pi}\right)^2} & k \text{ ungerade} \\
	\end{cases} \\
	\varphi_k &= \begin{cases}
		-\frac{\pi}{2} & k \text{ gerade} \\
		\pi + \arctan\left( \frac{\frac{1}{k \pi}}{-\frac{2}{k^2 \pi^2}} \right)& k \text{ ungerade}
	\end{cases}
\end{align*}


$A_k = \left| a_k \right|$, allerdings muss das angepasste Vorzeichen mit der Phase $\varphi$ korrigiert werden.


\subsection{«Schnelle» Berechnung von Fourier-Reihen}

Ermöglicht eine schnellere Berechnung, wenn gewisse Funktionseigenschaften vorhanden sind.

\begin{itemize}
\item Gerades Signal: Alle Sinus-Terme sind Null.
\item Integration über eine halbe Periode bei geraden Funktionen, beginnend bei 0. \\
		Multiplikation des ganzen mit 2.
\end{itemize}

\subsubsection{Multiplikation gerader und ungerader Funktionen:}

Ungerade: $-$, Gerade: $+$ dann gilt:
\begin{align*}
	 G \cdot G &= G \\
	 UG \cdot UG &= G \\
	 UG \cdot G &= UG \\
\end{align*}

\subsubsection{Gerade / Ungerade Funktion}

Sei $r(t)$ ein gerades Signal, dann ist $r(t) \cdot \cos(k \omega_1 t)$ gerade. $r(t) \cdot \sin(k \omega_1 t)$ ist ungerade und daher bei einem geraden Signal gleich $0$.

\begin{align*}
r(t) \text{ \space\space\space gerade }   &\Rightarrow b_k = 0 \\
r(t) \text{ ungerade } &\Rightarrow a_0 = 0 \land a_k = 0
\end{align*}

%TODO: Seite 31 Formeln abschreiben!


\subsection{Ableitung Fourier-Reihen aus Formelsammlung}

\subsubsection{Linearität}

Zwei Signale mit der selben Periode können leicht addiert werden ($s(t) = s_1(t) + s_2(t)$), indem alle Elemente jeweils addiert werden.

Wird ein Signal mit einer Konstante Multipliziert, können auch jeweils alle Komponenten in der Fourier-Reihe multipliziert werden.


\paragraph{Beispiel}

%TODO: Graph Beispiel 6 Demo Linearität

Bekannt: $T=2 \Rightarrow \omega_1 = \pi$

\begin{tabu}{l | l | l | l}
	$k$ & 0 & gerade & ungerade \\ \hline
	$a_k$ & 1/4 & 0 & $-\frac{2}{k^2\pi^2}$ \\ \hline
	$b_k$ & - & $-\frac{1}{k\pi}$ & $\frac{1}{k\pi}$
\end{tabu}


%TODO: Graph r(t) angeben


Gesucht: Fourierkoeffizienten von $r(t)$

$T=2$, $\omega_1 = \pi$

\subparagraph{Vorgehen:}
\begin{enumerate}
	\item Beziehung zwischen den Signalen aufstellen ($r(t)$ durch $s(t)$ ausdrücken): \[
			r(t) = 1-s(t)  
		\]
	\item Ersetzen der Formel aus 1. durch (bekannte) Fourierreihe und Resultat umformen, bis neue Fourier-Koeffizienten abgelesen werden können. (Tipp: $a_k$ und $b_k$ nicht ausformulieren, $\omega_1$ aber schon) \begin{align*}
			r(t) &= 1- \left( a_0 + \sum^\infty_{k=1} a_k \cos(k \pi t) + b_k \sin(k \pi t)\right) \\
			&= \underbrace{1 - a_0}_{\widehat{a_0}} + \sum^\infty_{k=1} \underbrace{-a_k}_{\widehat{a_k}} \cos(k \pi t) \underbrace{- b_k}_{\widehat{b_k}} \sin(k \pi t)
		\end{align*}
	\item Fourrierkoeffizienten von $r(t)$: 
	
	  \begin{tabu}{l | l | l | l}
			$k$ & 0 & gerade & ungerade \\ \hline
			$\widehat{a_k}$ & 3/4 & 0 & $\frac{2}{k^2 \pi^2}$ \\ \hline
			$\widehat{b_k}$ & - & $\frac{1}{k \pi}$ & $-\frac{1}{k \pi}$
		\end{tabu}
\end{enumerate}

\subparagraph{Amplituden-Phasenform}


\begin{align*}
	r(t) &= 1 - \left( A_0 + \sum^\infty_{k=1} A_k \cos(k \pi t - \varphi_k) \right) \\
		&= 1 - A_0 + \sum^\infty_{k=0} - A_k \cos(k \pi t - \varphi_k) \\
		&= 1 - A_0 + \sum^\infty_{k=1} + A_k \cos(k \pi t - (\varphi_k + \pi))
\end{align*}

(Wichtig: $\widehat{A_k}$ muss positiv sein!)

\begin{align*}
	\widehat{A_0} &= 1 - A_0 \\
	\widehat{A_k} &= A_k \\
	\widehat{\varphi_k} &= \varphi + \pi
\end{align*}


\subsubsection{Beispiel}


%TOOD: Signal von Auftrag 3 einbinden

\[
f(-t) = 2s(-t) -1
\]
\[
s(t) = \frac{1}{2} + \sum^\infty_{k=1} \left(-\frac{1}{k\pi}\right) \sin(2\pi k t)
\]

\begin{enumerate}
	\item Drücken Sie $r(t)$ durch $s(t)$ aus: \[
		r(t) = 2s(-t) -1 \left[ = \underbrace{-(2s(t) -1)}_{-f(t)} = 1 - 2s(t) \right]
		\]
	\item einsetzen von $s(t)$ und umformen \begin{align*}
		r(t) &= 2 \left( \frac{1}{2} + \sum^\infty_{k=1} \left(-\frac{1}{k\pi}\right) \sin(2\pi k (-t))\right) -1 \\
		&= 2 \sum^\infty_{k=1} - \frac{1}{k\pi} \sin(2\pi k (-t))\\
		&= \sum^\infty_{k=1} \frac{2}{k\pi} \sin(2 \pi k t)
	\end{align*}
\end{enumerate}

\subsection{Signalspiegelungen, Skalierungen und Verschiebungen}

Ein Signal $s(t)$ sowie deren Fourierreihe ist bekannt: \[
	s(t) = a_0 + \sum^\infty_{k=1} a_k \cos(k \omega_1 t) + b_k \sin(k \omega_1 t) = A_0 + \sum^\infty_{k=1} A_k \cos( k \omega_1 t - \varphi_k)
\]

\subsubsection{Spiegelung an $x$-Achse}
\[
	r(t) = -s(t) = -a_0 + \sum^\infty_{k=1} (-a_k) \cos(k \omega_1 t) + (-b_k) \sin(k \omega_1 t) = -A_0 + \sum^\infty_{k=1} \underbracket{A_k}_{\mathrlap{\text{Muss } >0}} \cos(k \omega_1 t - (\varphi_k + \pi))
\] \begin{align*}
	a_0 &\rightsquigarrow -a_0 \\
	a_k &\rightsquigarrow -a_k \\
	b_k &\rightsquigarrow -b_k \\
	\omega_1 &\rightsquigarrow \omega_1 \\
	A_0 &\rightsquigarrow -A_0 \\
	A_k &\rightsquigarrow A_k \\
	\varphi_k &\rightsquigarrow \varphi_k + \pi
\end{align*}

\subsubsection{Spiegelung an $y$-Achse (''Zeitumkehr'')}
\[
	r(t) = s(-t) = a_0 + \sum^\infty_{k=1} a_k \cos( k \omega_1 t) - b_k \sin(k \omega_1 t)
	     = A_0 + \sum^\infty_{k=1} A_k \cos(k\omega_1 t + \varphi_k)
\] \begin{align*}
a_0 &\rightsquigarrow a_0 \\
a_k &\rightsquigarrow a_k \\
b_k &\rightsquigarrow -b_k \\
\omega_1 &\rightsquigarrow \omega_1 \\
A_0 &\rightsquigarrow A_0 \\
A_k &\rightsquigarrow A_k \\
\varphi_k &\rightsquigarrow -\varphi_k
\end{align*}

\subsubsection{Streckung in $y$-Richtung um Faktor $c > 0$}
\[
	r(t) = c \cdot s(t) = c a_0 + \sum^\infty_{k=1} c a_k \cos(k \omega_1 t) + c b_k \sin(k \omega_1 t) = c A_0 + \sum^\infty_{k=1} c A_k \cos(k \omega_1 t - \varphi_k)
\] \begin{align*}
a_0 &\rightsquigarrow c a_0 \\
a_k &\rightsquigarrow c a_k \\
b_k &\rightsquigarrow c b_k \\
\omega_1 &\rightsquigarrow \omega_1 \\
A_0 &\rightsquigarrow c A_0 \\
A_k &\rightsquigarrow c A_k \\
\varphi_k &\rightsquigarrow \varphi_k
\end{align*}

\subsubsection{Streckung in $x$-Rightung um Faktor $c > 0$ (''Zeitskalierung'')}
\[
	r(t) = s(\underbracket{c}_{\mathclap{\text{Streckungsfaktor} \frac{1}{c}}} \cdot t) = a_0 + \sum^\infty_{k=1} a_k \cos(k \omega_1 c t) + b_k \sin(k \omega_1 c t)
		= A_0 + \sum^\infty_{k=1} A_k \cos(k \omega_1 c t - \varphi_k)
\] \begin{align*}
a_0 &\rightsquigarrow a_0 \\
a_k &\rightsquigarrow a_k \\
b_k &\rightsquigarrow b_k \\
\omega_1 &\rightsquigarrow c \omega_1 \\
A_0 &\rightsquigarrow A_0 \\
A_k &\rightsquigarrow A_k \\
\varphi_k &\rightsquigarrow \varphi_k \\
T &\rightsquigarrow \frac{1}{c} \cdot T
\end{align*}

\subsubsection{Verschiebung um $c$ entlang der $y$-Achse}
$r(t) = s(t) + c$; hat nur Einfluss auf $a_0$ und $A_0$

\subsubsection{Verschiebung um $c$ entlang der $x$-Achse}
Verschiebung um $c$ Einheiten nach rechts.
\begin{align*}
	r(t) &= s(t - c) = a_0 + \sum^\infty_{k=1} a_k \cos(k \omega_1 (t-c)) + b_k \sin (k \omega_1 (t - c)) \\
	&= a_0 + \sum^\infty_{k=1}
		a_k \left(\cos(k\omega_1 t)\cos(k\omega_1 c) + \sin(k \omega_1 t) \sin(k \omega_1 c)\right)
		+ b_k \left(\sin(k\omega_1 c) - \cos(k \omega_1 t) \sin(k \omega_1 c)\right) \\
	&= a_0 + \sum^\infty_{k=1}
		\underbracket{(a_k \cos(k \omega_1 c) - b_k \sin(k \omega_1 c))}_{\text{''neues'' } a_k} \cdot \cos(k \omega_1 t) +
		\underbracket{(a_k \sin(k \omega_1 c) + b_k \cos(k \omega_1 c))}_{\text{''neues'' } b_k} \cdot \sin(k \omega_1 t)
		%TODO: This has yet to come: Amplituden-Phasen-Form
\end{align*} \begin{align*}
a_0 &\rightsquigarrow a_0 \\
a_k &\rightsquigarrow a_k \cos(k \omega_1 c) - b_k \sin(k \omega_1 c) \\
b_k &\rightsquigarrow a_k \sin(k \omega_1 c) - b_k \cos(k \omega_1 c) \\
\omega_1 &\rightsquigarrow c \omega_1 \\
A_0 &\rightsquigarrow A_0 \\
A_k &\rightsquigarrow A_k %\\
%\varphi_k &\rightsquigarrow \varphi_k
%TODO: This has yet to come: Amplituden-Phasen-Form
\end{align*}

\subsubsection{Beispiel}

%TODO: Graphik Signal aus Beispiel 7

\[
	r(t) = \frac{\pi}{4} - \frac{1}{\pi} \cos(t) + \frac{1}{2} \sin(t) + ...
\]

\[
	s(t) =^1 \frac{1}{\pi} r( \pi \cdot t) =^2 \frac{1}{\pi} \left( \frac{\pi}{4} \cos(\pi t) + \frac{|}{2} \sin(\pi t) + ... \right) = \underbracket{\frac{1}{4}}_{a_0} - \underbracket{- \frac{1}{\pi^2}}_{a_1} \cos(\pi t) + \underbracket{\frac{1}{2 \pi}}_{b_1} \sin(\pi t) + ...
\]

\begin{enumerate}
	\item $s(t)$ durch $r(t)$ ausdrücken
	\item einsehen und vereinfachen
	\end{enumerate}


\subsubsection{Beispiel II}
%TODO: Signal gem Übung XXXXX

\[
s(t) = \frac{1}{4} + \frac{1}{\pi} \sum^\infty_{k=1} \frac{1}{k} \left[ \sin\left(\frac{k \pi}{2}\right) \cos(k t) + \left(1 - \cos\left(\frac{k \pi}{2}\right)\right) \sin(kt) \right]
\]

Wiederholt sich nach $k=4 \approx k=0$

\paragraph{Vorgehen:}
\begin{enumerate}
	\item Strecken der Ausgangsfunktion um $\frac{\pi}{2}$: $ \tilde{s}(t) = s\left(\frac{\pi}{2} t\right)$
	\item Verschieben um $1$: \[
		r(t) = \tilde{s}(t-1) = s\left(\frac{\pi}{2}(t-1)\right) = s\left(\frac{\pi}{2}t - \frac{\pi}{2}\right)
	\]
	\item ''Ausklammern'' $a_k$ und $b_k$, einsetzen neues $s(t)$: \begin{align*}
	a_k &= \frac{1}{k\pi} \sin\left(\frac{k\pi}{2}\right) \\
	b_k &= \frac{1}{k\pi} \left(1 - \cos\left(\frac{k\pi}{2}\right)\right) \\
	r(t) &= \frac{1}{4} + \sum^\infty_{k=1} a_k \cos\left(k \left(\frac{\pi}{2}t
		- \frac{\pi}{2} \right)\right)
		+ b_k \sin\left(k\left(\frac{\pi}{2} t
		- \frac{\pi}{2}\right)\right) \\
	&= \frac{1}{4} + \sum^\infty_{k=1} a_k \cos\left(\frac{k\pi}{2} t - \frac{k \pi}{2}\right) + b_k \sin\left( \frac{k\pi}{2} t - \frac{k\pi}{2}\right) \\
	&= \frac{1}{4} + \sum^\infty_{k=1}
		a_k \left(\cos\left(\frac{k\pi}{2} t\right)\cos\left(\frac{k\pi}{2}\right) + \sin\left(\frac{k\pi}{2} t\right)\sin\left(\frac{k\pi}{2}\right) \right) \\
		&\hspace{1.35cm} + b_k \left(\sin\left(\frac{k\pi}{2} t\right)\cos\left(\frac{k\pi}{2}\right) - \cos\left(\frac{k\pi}{2} t\right)\sin\left(\frac{k\pi}{2}\right)\right) \\
	&= \frac{1}{4} + \sum^\infty_{k=1}
		\underbrace{\left( a_k \cos\left(\frac{k\pi}{2}\right) - b_k \sin\left(\frac{k\pi}{2}\right) \right)}_{\hat{a}_k} \cos\left(k\frac{\pi}{2} t\right) 
		+ \underbrace{\left( a_k \sin\left(\frac{k\pi}{2}\right) + b_k \cos\left(\frac{k\pi}{2}\right) \right)}_{\hat{b}_k} \sin\left(k\frac{\pi}{2} t\right) \\
	\hat{a}_0 &= \frac{1}{4} \\
	\hat{a}_k &= \text{siehe oben} \\
	\hat{b}_k &= \text{siehe oben} \\
	\hat{\omega}_1 &= \frac{\pi}{2}
	\end{align*}

\item Komponenten berechnen (hier exemplarisch für $\hat{a}_k$): 
	\[
	\hat{a}_k = \frac{1}{k\pi} \sin\left(\frac{k\pi}{2}\right) \cos\left(\frac{k\pi}{2}\right) - \frac{1}{k\pi}\left(1- \cos\left(\frac{k\pi}{2}\right) \sin\left(\frac{k\pi}{2}\right)\right)
		\]
		Möglichkeiten: \begin{itemize}
			\item Durch Trigo ableiten / vereinfachen.
			\item Alle Möglichkeiten für verschiedene Zahlen gem. kleinstem gemeinsamen Zyklus auflisten, $k$ entsprechend einsetzen: \[
			\hat{a}_k = \begin{cases}
				\frac{1}{k\pi} (0 \cdot 1 - (1 - 1) \cdot 0) = 0 & \text{für } k=0 mod 4 \\
				\frac{1}{k\pi} (1 \cdot 0 - (1 - 0) \cdot 1) = -\frac{1}{k\pi} & \text{für } k=1 mod 4 \\
				\frac{1}{k\pi} (0 \cdot -1 - (1 + 1) \cdot 1) = 0 & \text{für } k=2 mod 4 \\
				\frac{1}{k\pi} (-1 \cdot 0 - (-1 -0) \cdot -1) = \frac{1}{k\pi} & \text{für } k=3 mod 4
			\end{cases}
			\]
		\end{itemize}
\end{enumerate}

\subsection{Anwedungen}

Moving Average; ''Glättung'' von Signalen mit rauschen


\end{document}