% This is a default-selection of plugins that are used widely in this repo.

\documentclass[a4paper,10pt,fleqn]{article}
\usepackage[utf8]{inputenc}

% deutsche Trennmuster etc.
\usepackage[ngerman]{babel}
\usepackage[T1]{fontenc}

% mathematical simbols and fonts
\usepackage{mathtools} 
\usepackage{amssymb}
\usepackage{amsmath}
\usepackage{ntheorem}
\usepackage{polynom}
\usepackage{marvosym}
\usepackage{tabu}
\renewcommand*{\bmod}{\mathbin{\%}}
\everymath{\displaystyle}

\usepackage{multicol}
\usepackage{color}
\usepackage[usenames,dvipsnames]{xcolor}
\setlength{\columnsep}{1cm}
\setlength{\columnseprule}{0.25pt}
\def\columnseprulecolor{\color{gray}}
\usepackage{hyperref}

\usepackage[margin=1.5cm]{geometry}
\usepackage{graphicx}
\usepackage{pgfplots}
\pgfplotsset{compat=1.10}

%Code higlighting

\usepackage{minted}

% make lists more compact:
\newlength{\wideitemsep}
\setlength{\wideitemsep}{.5\itemsep}
\addtolength{\wideitemsep}{-5pt}
\let\olditem\item
\renewcommand{\item}{\setlength{\itemsep}{\wideitemsep}\olditem}
\renewcommand{\arraystretch}{1.25}

\title{Zusammenfassung An2I}
\author{Fabian Hauser}
 
\begin{document}
\maketitle

\section{Taylorreihen}
Jede ableitbare Funktion lässt sich durch Polynome approximieren.


\paragraph{Beispiel eines Polynoms 3. Ordnung}
\begin{align*}
	p(x_0) & = a(x_0-x_0)^3 + b(x_0-x_0)^2 + c(x_0-x_0) + 0 &= d = 0!d = 1d \\
	p'(x_0) &= 3a \cdot (x_0-x_0)^2 + 2b * (x_0-x_0) + c + 0 &= 1 \cdot c = 1!c = 1c \\
	p''(x_0) &= 2 \cdot 3a (x_0-x_0) + 2b + 0 + 0 &= 1 \cdot 2b = 2!b = 2b \\
	p^{(3)}(x_0) &= 2 \cdot 3a + 0 + 0 + 0 &= 1 \cdot 2 \cdot 3a = 3!a = 6a \\
	p^{(4)}(x_0) &= 0 + 0 + 0 + 0 &= 0
\end{align*}
was bedeutet:
\begin{align*}
	d &= \dfrac{p(x_0)}{0!} \\
	c &= \dfrac{p'(x_0)}{1!} \\
	b &= \dfrac{p''(x_0)}{2!} \\
	a &= \dfrac{p^{(3)}(x_0)}{3!}
\end{align*}

Wenn wir folgendes Polynom vervollständigen möchten:
\begin{align*}
	p(x) &= 2x^3 - 5x + 7 = [\frac{12}{3!}](x-1)^3 + [\frac{12}{2!}](x-1)^2 + [\frac{12}{1!}](x-1) + [\frac{4}{0!}] \\
	p'(x) &= 6x^2 - 5 => p'(1) = 6 * 1^2 -5 ) =1 \\
	p''(x) = 12x => p''(1) = 12 \\
	p^{(3)}(x) = 12 => p^{(3)}(1) = 12
\end{align*}
(es wird jeweils das Polynom an der Position 1 entwickelt)

\paragraph{Beispiel II}

Ausgangslage: Funktion $f(x)$, Entwicklungspunkt $x_0$, Ordnung $\mathbb{N}$ \\

Taylorpolynom der Ordnung $\mathbb{N}$ : 
%TODO: Cleanup next equation
\begin{align*}
P_N(x) &= f(x_0)/0! + f'(x_0)/1! * (x - x_0) = f''(x_x0) / 2! * x(-x_0)^2 + ... + f^(n)(x_0)/n! (x-x_0)^n \\
&= \sum^N_{k=0}{\frac{k^{(k)}(x_0)}{k!} (x-x_0)^k}
\end{align*}

Übrigens: Ein Taylorpolynom der 1. Ordnung ist die Linearisierung:
\begin{align*}
	f(x_0) + f'(x_0)(x-x_0)
\end{align*}

Feststellung: für $x \approx x_0$ gilt $f(x) \approx P_N(x)$, je grösser N, umso besser.


\paragraph{Beispiel III}

\begin{align*}
	f(x) &= e^x & \text{Entwicklungspunkt: } x_0=0 \\
	f'(x) &= e^x &= f(0) = 1\\
	f^{(n)}(x) &= e^x &= f^{(n)}(0) = 1
\end{align*}
was im Taylorplynom resultiert:
\begin{align*}
e^x \approx \frac{1}{0!} + \frac{1}{1!}(x-0)^1 + \frac{1}{2!}(x-0)^2 + \frac{1}{3!}(x-0)^3 + ...
\end{align*}

Eine Taylorreihe ist ein Taylerpolynom der $\infty$ Ordnung, und genau gleich wie die Ursprungsfunktion.

\emph{Konvergenzradius}: Distanz vom entwicklungspnukt, bis den das Taylorpolynom gilt. Im Fall von $e^x$ ist dieser $\infty$ gross.

\subsection{Fehlerabschätzung}

Gesucht: $f(x)$, Entwicklungspunkt $x_0$, Ordnung N des Taylorpolynoms:

\[
	P_N(x) = \sum^{N}_{k=0}{\frac{f^{(k)}(x_0)}{k!} (x-x_0)^k}
\]

Rechenfehler: $\left| f(x) - P_N(x) \right| \leq \frac{m}{N+1)} \left| x-x_0 \right|^{N+1}$ mit $m \geq \left| f^{(N+1)}(t) \right|$

$t$: Punkt zwischen $x$ und $x_0$, an dem $\left| f^{(N+1)}(t) \right|$ genau der Differenz entspricht.

Ziel: $f(x)$ im Interfavv $[a,b]$ mit maximalem fehler $\Delta$ zu bestimmen.
Suche Zahl $m \leq$ max $f^{(N+1)}(x)$ für $x \in [a,b]$
%TODO: Siehe Script, vervollständigen

\paragraph{Beispiel}

Ziel: Sinus im Intervall $[0, \pi]$ mit Genauigkeit $\Delta = \frac{1}{100}$

\begin{align*}
	f(x) &= sin(x) \\
	f'(x) &= cos(x) \\
	f^{(2)} &= -sin(x) \\
	f^{(3)} &= -cos(x) \\
	f^{(4)} &= sin(x) \\
\end{align*}

Beim Sinus gilt:
\[
	\left| f^{(N+1)}(x) \right| \leq m = 1
\]

\[
	\left| sin(x) - P_N(x) \right| \leq \frac{1}{(N+1)^1} \left(\frac{\pi-0}{2}\right)^{N+1} < \frac{1}{100}
\]


\end{document}