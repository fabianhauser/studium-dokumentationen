% This is a default-selection of plugins that are used widely in this repo.

\documentclass[a4paper,10pt,fleqn]{article}
\usepackage[utf8]{inputenc}

% deutsche Trennmuster etc.
\usepackage[ngerman]{babel}
\usepackage[T1]{fontenc}

% mathematical simbols and fonts
\usepackage{mathtools} 
\usepackage{amssymb}
\usepackage{amsmath}
\usepackage{ntheorem}
\usepackage{polynom}
\usepackage{marvosym}
\usepackage{tabu}
\renewcommand*{\bmod}{\mathbin{\%}}
\everymath{\displaystyle}

\usepackage{multicol}
\usepackage{color}
\usepackage[usenames,dvipsnames]{xcolor}
\setlength{\columnsep}{1cm}
\setlength{\columnseprule}{0.25pt}
\def\columnseprulecolor{\color{gray}}
\usepackage{hyperref}

\usepackage[margin=1.5cm]{geometry}
\usepackage{graphicx}
\usepackage{pgfplots}
\pgfplotsset{compat=1.10}

%Code higlighting

\usepackage{minted}

% make lists more compact:
\newlength{\wideitemsep}
\setlength{\wideitemsep}{.5\itemsep}
\addtolength{\wideitemsep}{-5pt}
\let\olditem\item
\renewcommand{\item}{\setlength{\itemsep}{\wideitemsep}\olditem}
\renewcommand{\arraystretch}{1.25}

\title{Zusammenfassung An2I}
\author{Fabian Hauser}
 
\begin{document}
\maketitle

\section{Taylorreihen}
Jede ableitbare Funktion lässt sich durch Polynome approximieren.


\paragraph{Beispiel eines Polynoms 3. Ordnung}
\begin{align*}
	p(x_0) & = a(x_0-x_0)^3 + b(x_0-x_0)^2 + c(x_0-x_0) + 0 &= d = 0!d = 1d \\
	p'(x_0) &= 3a \cdot (x_0-x_0)^2 + 2b * (x_0-x_0) + c + 0 &= 1 \cdot c = 1!c = 1c \\
	p''(x_0) &= 2 \cdot 3a (x_0-x_0) + 2b + 0 + 0 &= 1 \cdot 2b = 2!b = 2b \\
	p^{(3)}(x_0) &= 2 \cdot 3a + 0 + 0 + 0 &= 1 \cdot 2 \cdot 3a = 3!a = 6a \\
	p^{(4)}(x_0) &= 0 + 0 + 0 + 0 &= 0
\end{align*}
was bedeutet:
\begin{align*}
	d &= \dfrac{p(x_0)}{0!} \\
	c &= \dfrac{p'(x_0)}{1!} \\
	b &= \dfrac{p''(x_0)}{2!} \\
	a &= \dfrac{p^{(3)}(x_0)}{3!}
\end{align*}

Wenn wir folgendes Polynom vervollständigen möchten:
\begin{align*}
	p(x) &= 2x^3 - 5x + 7 = [\frac{12}{3!}](x-1)^3 + [\frac{12}{2!}](x-1)^2 + [\frac{12}{1!}](x-1) + [\frac{4}{0!}] \\
	p'(x) &= 6x^2 - 5 => p'(1) = 6 * 1^2 -5 ) =1 \\
	p''(x) = 12x => p''(1) = 12 \\
	p^{(3)}(x) = 12 => p^{(3)}(1) = 12
\end{align*}
(es wird jeweils das Polynom an der Position 1 entwickelt)

\paragraph{Beispiel II}

Ausgangslage: Funktion $f(x)$, Entwicklungspunkt $x_0$, Ordnung $\mathbb{N}$ \\

Taylorpolynom der Ordnung $\mathbb{N}$ : 
%TODO: Cleanup next equation
\begin{align*}
P_N(x) &= f(x_0)/0! + f'(x_0)/1! * (x - x_0) = f''(x_x0) / 2! * x(-x_0)^2 + ... + f^(n)(x_0)/n! (x-x_0)^n \\
&= \sum^N_{k=0}{\frac{k^{(k)}(x_0)}{k!} (x-x_0)^k}
\end{align*}

Übrigens: Ein Taylorpolynom der 1. Ordnung ist die Linearisierung:
\begin{align*}
	f(x_0) + f'(x_0)(x-x_0)
\end{align*}

Feststellung: für $x \approx x_0$ gilt $f(x) \approx P_N(x)$, je grösser N, umso besser.


\paragraph{Beispiel III}

\begin{align*}
	f(x) &= e^x & \text{Entwicklungspunkt: } x_0=0 \\
	f'(x) &= e^x &= f(0) = 1\\
	f^{(n)}(x) &= e^x &= f^{(n)}(0) = 1
\end{align*}
was im Taylorplynom resultiert:
\begin{align*}
e^x \approx \frac{1}{0!} + \frac{1}{1!}(x-0)^1 + \frac{1}{2!}(x-0)^2 + \frac{1}{3!}(x-0)^3 + ...
\end{align*}

Eine Taylorreihe ist ein Taylerpolynom der $\infty$ Ordnung, und genau gleich wie die Ursprungsfunktion.

\emph{Konvergenzradius}: Distanz vom entwicklungspnukt, bis den das Taylorpolynom gilt. Im Fall von $e^x$ ist dieser $\infty$ gross.

\subsection{Fehlerabschätzung}

Gesucht: $f(x)$, Entwicklungspunkt $x_0$, Ordnung N des Taylorpolynoms:

\[
	P_N(x) = \sum^{N}_{k=0}{\frac{f^{(k)}(x_0)}{k!} (x-x_0)^k}
\]

Rechenfehler: $\left| f(x) - P_N(x) \right| \leq \frac{m}{N+1)} \left| x-x_0 \right|^{N+1}$ mit $m \geq \left| f^{(N+1)}(t) \right|$

$t$: Punkt zwischen $x$ und $x_0$, an dem $\left| f^{(N+1)}(t) \right|$ genau der Differenz entspricht.

Ziel: $f(x)$ im Intervall $[a,b]$ mit maximalem Fehler $\Delta$ zu bestimmen.
Suche Zahl $m \leq$ max $f^{(N+1)}(x)$ für $x \in [a,b]$. $a$ und $b$ ist der mögliche Wertebereich der Funktion $f(x)$
%TODO: Siehe Script, vervollständigen

\subsubsection{Abschätzung}



\paragraph{Beispiel}

Ziel: Sinus im Intervall $[0, \pi]$ mit Genauigkeit $\Delta = \frac{1}{100}$

\begin{align*}
	f(x) &= sin(x) \\
	f'(x) &= cos(x) \\
	f^{(2)} &= -sin(x) \\
	f^{(3)} &= -cos(x) \\
	f^{(4)} &= sin(x) \\
\end{align*}

Beim Sinus gilt:
\[
	\left| f^{(N+1)}(x) \right| \leq m = 1
\]

\[
	\left| sin(x) - P_N(x) \right| \leq \frac{1}{(N+1)^1} \left(\frac{\pi-0}{2}\right)^{N+1} < \frac{1}{100}
\]

\subsection{Linearisieren / Taylorpolynom 1. Ordnung)}
Gesucht: $f(x), x_0$: Ziel Linearisierung um $x_0$

Antwort: $f(x) \approx f(x_0) + f'(x_0)(x-x_0) = P_1(x)$

In welchem Intervall ist der Rechenfehler $< \Delta$?

\[
	\left|f(x) - p_1(x) \right| \leq \frac{m}{2!}(?)^2 < \Delta \\
\]

\[
	|?| < \sqrt{\frac{2 \Delta}{m}}
\]

Antwort für $x \in \left[ x_0 - \sqrt{\frac{2 \Delta}{m}}; x_0 + \sqrt{\frac{2 \Delta}{m}} \right] = I_m$

Was ist also $m$? Globales Maximum von $|f''(x)|$ im Intervall $I_m$
Problem: es gibt eine Rückkopplung. \\
Angenommen werden kann ein Wert von ca. $|f''(\text{ca. } 1.5 \cdot x_0)|$


\subsubsection{Beispiel}
	EP: $x_0 = 1$
	
\begin{align*}
	f(x) &= sin(x^2) &\rightarrow f(1) = sin(1) \\
	f'(x) &= cos(x^2) \cdot 2x &\rightarrow f'(1) = 2 cos(1) \\
	f''(x) &= -sin(x^2) \cdot (2x)^2 + cos(x^2) \cdot 2 &\rightarrow f''(1) = 2 cos(1) - 4 sin(1)
\end{align*}

\[
	f(x) \approx sin(1) + 2 cos(1) (x-1) = P_1(x) 
\]

Da $|f(x) - p_1(x)| < \Delta$, gilt das Intervall für $x \in \left[1-\sqrt{\frac{2 \Delta}{m}}; 1+\sqrt{\frac{2 \Delta}{m}} \right] = I$.

\[
	f''(1) \approx 2,29
\]
Hypothese: $m=3 \approx 1.5 \cdot 2.29$

\[
	I = \left[1-\sqrt{\frac{0.002 \Delta}{3}}; 1+\sqrt{\frac{0.002 \Delta}{3}} \right]
\]

Test: $|f''(x)| \leq 3$ für all $x \in I$ ?

\subsection{Konvergenzradius}

Bereich um den Entwicklungspunkt, in dem die Taylorreihe den richtigen Wert der Ausgangsfunktion liefert.

Üblicherweise ist der Konvergenzradius einer Taylorreihe gleich dem Abstand vom Entwicklungspunkt zur ersten Definitionslücke (in den Reelen oder Komplexen Zahlen).

\section{Grenzwerte}

\subsection{Eigentliche Endwerte im Unendlichen}
Grenzwert ist gleich Null: $f(x) \xrightarrow{x \rightarrow \infty} 0$

Limes: $\lim_{x \rightarrow \infty}{f(x)} = 0$

Uneigentliche Grenzwerte sind $\infty$, was aber streng genommen keine Zahl ist.

\paragraph{Beispiele:}

\begin{align*}
	\lim_{x \to \infty}{arctan(x)} &= \frac{\pi}{2} \\
	\lim_{x \to -\infty}{arctan(x)} &= -\frac{\pi}{2}
\end{align*}

\subsubsection{Systematische Analyse}

der Funktion $f(x): \frac{\sin{(x)}}{x}$. Behauptung: $ \lim_{x \to \infty}{f(x)} = 0$

Streifen um den angenommenen Grenzwert 0. Der "Radius" wird als $\epsilon$ bezeichnet.

\paragraph{Grenzzahl $X(\epsilon)$,} ab dem der Funktionswert innerhalb dieser Zahl liegt. Wenn es keine Rolle spielt, wie gross $\epsilon$ ist, um ein Resultat zu liefern, stimmt der angenommene Grenzwert.

\paragraph{Beweis (Definition Grenzwert):} Für jede noch so kleine Zahl $\epsilon > 0$ liegt die Funktion ab einer Stelle $X$ (bzw. $X(\epsilon)$) vollständig in einem Streifen der vom Grenzwert $0$ um jeweils ein $\epsilon$ nach oben und unten weggeht.
\[
\text{D.h.:  alle } x > X(\epsilon) \text{ gilt: } |f(x)-0| < \epsilon
\]
Auf jeden Fall gilt, $X = \frac{1}{\epsilon}$

Trick zur Grenzwertanalyse bei z.B. $x^2$: Umkehren zu $\frac{1}{x^2}$, was qualitativ gleich aussieht wie $\frac{1}{x}$

\subsection{Eigentliche Grenzwerte im Endlichen}

\paragraph{Beispiel:} von $\lim\limits_{x \to 2}{\frac{x^2 + x -6}{x-2}}$.

Vorgehen: Stetige Fortsetzung suchen, Zahlenwert bei Definitionslücke  einsetzen.


Wenn die Funktion an der gesuchten Stelle definiert ist, wird dieser wert daraus entfernt. So wird zum Beispiel aus $\lim\limits_{x \to 0}{sig^2(x)} = 1$ wird $\lim\limits_{x \to 0}{\text{sig}^2_{|\mathbb{R} \setminus \{0\}}(x)}$

\subsection{Einseitige Grenzwerte}

	Beachtet nur ein Teilbereich der Funktion; Beim rechtsseitige Grenzwert werden die kleineren Zahlen ignoriert:
\[
	\lim\limits_{x \to 0+}{\text{sig}(x)} = \lim\limits{x \to 0} \text{sig}|_{\mathbb{R} \setminus [0;\infty)}(x) = 1
\]
bzw. die grösseren Zahlen ignoriert:
\[
	\lim\limits_{x \to 0-}{\text{sig}(x)}
\]

\subsubsection{Rechentechniken}

\paragraph{Subfunktionen}

\paragraph{Unstetige Stellen}

\[
	\lim\limits_{x \to 0}(x \cdot \text{sig}(x)) = 0
\]

$x$ geht gegen Null; sig$(x)$ \emph{existiert nicht} an der Stelle 0. Ein möglicher Weg:
\[
	\lim\limits_{x \to 0+}(x \cdot \text{sig}(x)) = 0
\]

$x$ ist nach wie vor 0; Beim Signum können wir 1 annehmen (da positive Zahl). Beim umgekehrten Weg:
\[
	\lim\limits_{x \to 0-}(x \cdot \text{sig}(x)) = 0
\]

können wir $x$ als 0 annehmen; Beim Signum können wir $-1$ annehmen (da negative Zahl).

Der beidseitige Grenzwert ist gleich; daraus können wir schliessen, dass der Grenzwert auch $0$ ist.

In diesem Fall wäre auch die Analyse der eigentlichen Funktion; in diesem Fall steht nur ein komplizierter Betrag.

\paragraph{Gegen unendlich}

\[
	\lim\limits{x \to \infty} \sin(\frac{\pi x^2 + 3x - 4}{4x^2-4})
\]
Zähler und Nenner bewegen sich gegen unendlich. Lösungsansatz: Schnellst anwachstende Komponente ausklammern:

\[
	\lim\limits_{x \to infty} \sin(\frac{x^2(\pi + \frac{3}{x} - \frac{4}{x^2})}{x^2(4 - \frac{4}{x^2})}) =
	\lim\limits_{x \to infty} \sin(\frac{\pi + \frac{3}{x} - \frac{4}{x^2}}{4 - \frac{4}{x^2}}) = sin(\frac{\pi}{4}) = \frac{\sqrt{2}}{2}
\]


\subsection{Uneigentliche Grenzwerte}

$\infty$ und $-\infty$ zerstören die üblichen Zahlenkörper; deshalb gelten damit die normalen Rechengesetze nicht mehr. Deshalb: Unendlich ist eher Sammelbegriff für "sehr viel", aber es ist nicht genau klar, "wie viel".


\subsubsection{Beispiele}
Sehr grosse Zahl in Exponentialfunktion einsetzten, ergibt eine sehr grosse Zahl.
\[
	\lim\limits_{x \to \infty} e^x = \infty
\]

Konstante $C$, über welcher die Funktion am Ort $X(C)$ ist.

In diesem Fall: Für alle $x > \ln(x)$ gilt $x^x > e^{\ln(C)} = C$


\subsubsection{$\infty$}

Unklar: $\infty - \infty$
B/L: $\infty \cdot (\pm 0)$; $\frac{\infty}{\infty}$; $\frac{0\pm}{0\pm}$

\[
	\lim\limits_{ x \to \infty} \frac{x^2-3x -7}{\sqrt{11x^4+3x+9}} \left(\approx \lim\limits_{ x \to \infty} \frac{(\infty - \infty) - 7}{\infty}\right) =
	 \lim\limits_{ x \to \infty} \frac{x^2 (1 - \frac{3}{x} - \frac{7}{x^2}}{\sqrt{11x^4 + 3x + 9}} \left(\approx \lim\limits_{ x \to \infty} \frac{\infty}{\infty}\right) =
	 \lim\limits_{x \to \infty} %TODO: Multiplikation mit 1/x^2
	 = \frac{1}{\sqrt{11}}
\]

\subsection{Regel von Bernoulli / Lobital}

\[
	\lim\limits_{x \to q} \frac{f(x)}{g(x)} =^{\text{Typ } \frac{0}{0}} \lim\limits_{x \to q} \frac{f'(x)}{g'(x)} %TODO: =-text anpassen
\]


\subsection{Ableitung mit dem limes}

\[
	f'(x) = \frac{df}{dx}|_{x=x_0} = \lim\limits_{x \to x_0} \frac{f(x) - f(x_0)}{x - x_0} =^\frac{0}{0} = \lim\limits_{x \to x_0} \frac{\frac{d}{dx}(f(x) - f(x_0))}{\frac{d}{dx}(x - x_0} = \lim\limits{x \to x_0} \frac{f'(x) -0}{1 - 0} = f'(x_0)
\]

\section{Integral}

\subsection{Stammfunktionen}

Umkehrung der Ableitung.

Gegeben: Funktion $f(x)$

Gesucht: Funktion $F(x)$ (da Stammfunktion, stetige Funktion)

Eigenschaft: $F'(x) = f(x)$ für all $x$ (aus dem Definitionsbereich von $x$ in $f(x)$)

Dann heisst $F$ Stammfunktion von $f$.

\paragraph{Beispiel} \hfill

$f(x) = \sin(x)$

$F(x) = -\cos(x) + c$

Vergewissern: $F'(x) = \frac{\mathrm{d}}{\mathrm{d}x}(\cos(x) + c) = -\sin(x) = f(x)$

Mögliches vorgehen: «dirty» Ableitung rückgängig machen, prüfen, ob sie beim Ableitung richtig rauskommen, nachkorrigieren.

\subsection{Unbestimmtes Integral}

Das Unbestimmte Integral dient immer der Suche nach der Stammfunktion.
Sei $f(x)$ eine Funktion und $F(x)$ eine Stammfunktion von $f$.

 \[
 \int \underbracket{f(x)}_{\mathclap{\text{Integrand}}}\, \overbracket{\mathrm{d}x}^{\mathclap{\text{Integrationsvariable}}} = F(x) + \underbracket{\text{ const}}_{\mathclap{\text{Integrationskonstante}}}
 \]
 
 ist $\Leftrightarrow$ zu:
 
 \[
	 f(x) = \frac{\mathrm{d}}{\mathrm{d}x}(F(x + \text{ const}))
 \]


$+\text{const}$ bedeutet: Die Zahlen $a + const = b + const$ sind $a$ und $b$ in der gleichen Restklasse (d.h. es gibt einen Modulo-Konstante.)



\subsection{Bestimmtes Integral}


\end{document}