% This is a default-selection of plugins that are used widely in this repo.

\documentclass[a4paper,10pt,fleqn]{article}
\usepackage[utf8]{inputenc}

% deutsche Trennmuster etc.
\usepackage[ngerman]{babel}
\usepackage[T1]{fontenc}

% mathematical simbols and fonts
\usepackage{mathtools} 
\usepackage{amssymb}
\usepackage{amsmath}
\usepackage{ntheorem}
\usepackage{polynom}
\usepackage{marvosym}
\usepackage{tabu}
\renewcommand*{\bmod}{\mathbin{\%}}
\everymath{\displaystyle}

\usepackage{multicol}
\usepackage{color}
\usepackage[usenames,dvipsnames]{xcolor}
\setlength{\columnsep}{1cm}
\setlength{\columnseprule}{0.25pt}
\def\columnseprulecolor{\color{gray}}
\usepackage{hyperref}

\usepackage[margin=1.5cm]{geometry}
\usepackage{graphicx}
\usepackage{pgfplots}
\pgfplotsset{compat=1.10}

%Code higlighting

\usepackage{minted}

% make lists more compact:
\newlength{\wideitemsep}
\setlength{\wideitemsep}{.5\itemsep}
\addtolength{\wideitemsep}{-5pt}
\let\olditem\item
\renewcommand{\item}{\setlength{\itemsep}{\wideitemsep}\olditem}
\renewcommand{\arraystretch}{1.25}

\title{Zusammenfassung DMI - Diskrete Mathematik für Informatiker}
\author{Fabian Hauser}
 
\begin{document}
\maketitle
\begin{multicols}{2}
[
\section{Logik}
]

	Bei DMI ist das Ausgedruckte Script mit Notizen auf dem Skript zur Prüfung zugelassen.

\subsection{Operatoren}

%TODO: Hier fehlt noch was. Papier!

\subsubsection{NAND (Nicht UND)}

$A \uparrow B = \overline{A \land B}$

\subsubsection{Teiler von}
	$x$ ist Teiler von $n$: $x|n$

\subsubsection{Äquivalenz}
	Etwas ist genau gleich.
	\[
	t \Leftrightarrow t, f \Leftrightarrow f
	\]y

\subsubsection{Implikation}
	Wenn a, dann b.
	
	\begin{tabular}{| c | c | c |}
	\hline 
	$A$ & $B$ & $A \Rightarrow B$ \\
	\hline
	w & w & w \\
	w & f & f \\
	f & w & w \\
	f & f & w \\
	\hline
	\end{tabular}



\subsubsection{Prädikatenlogik}
	\begin{tabular}{| l | l | l |}
		\hline
		$\forall$ & Allquantor & Für alle $x$ in $M$ gilt, \\ \hline
		$\exists$ & Existenzquantor &
		Es gibt $x$ in $M$, für \\
		& & die gilt, \\
		\hline
	\end{tabular}

	\paragraph{Beispiele}
		\begin{align*}
		A(n): & 5 | n \\
		B(n): & 1+2+3+\dots+n = \sum_{k=1}^{m}k = \frac{1}{2} n (n+1) \\
		R(x): & \text{der Weg X führt nach Rom}
		\end{align*}
		
		Alle Wege führen nach Rom:
		\[
		\forall x \in W \underbracket{:}_\text{gilt} R(x)
		\]
		
		Nicht alle Wege führen nach Rom $\Leftrightarrow$ Es gibt einen Weg, der nicht nach Rom führt:
		
		$\neg \left(\forall X \in W : R\left(X\right)\right) \Leftrightarrow \exists X \in W : \neg R(x)$
	

\subsubsection{Summenzeichen}

	$ \sum_{k=1}^{3} k = 1+2+3 = 6$

\subsection{Rechnen mit Logischen Aussagen}

Als eine \emph{Tautologie} wird eine Aussage bezeichnet, welche immer Wahr ist.

	
\subsubsection{Verknüpfungen}

	\begin{description}

		\item[Bindungsstärke] \hfill \\$\neg$ vor $\lor$, $\land$ vor $\Leftrightarrow$, $\Rightarrow$
		\item[De Morgan]\hfill \\ $\neg (A \land B) \Leftrightarrow \neg A \lor \neg B$ \newline $\neg (A \lor B) \Leftrightarrow \neg A \land \neg B$
		\item[Kommutativ] hfill \\ $A \land B \Leftrightarrow B \land A$ $A \lor B \Leftrightarrow B \lor A$
		\item[Assoziativ] \hfill \\ $A \lor (B \lor C) \Leftrightarrow (A \lor B) \lor C$
		\item[Distributiv] \hfill \\ $A \land (B \lor C) \Leftrightarrow (A \land B) \lor (A \land C)$ \newline $A \lor (B \land C) \Leftrightarrow (A \lor B) \land (A \lor C)$ 

	\end{description}


\subsubsection{Normalformen}
	Normalformen ist die Darstellung einer Logischen Aussage, wie sie aus einer Wahrheitstafel abgelesen werden kann.
	
	\paragraph{Disjunktive Normalform}
		
		Verknüpfung der logischen Ausdrücke mit $\lor$
		\[
			(X \land \neg Y \land Z) \lor
			(X \land Y \neg \land \neg Z) \lor
			(\neg X \land Y \land \neg Z) \lor
			(\neg X \land \neg Y \land \neg Z)
		\]
	
	\paragraph{Konjunktive Normalform}
		
		Verknüpfung der logischen Ausdrücke mit $\land$
		\[
			(\neg X \lor \neg Y \lor \neg Z) \land
			(\neg X \lor \neg Y \lor Z) \land
			(X \lor Y \lor \neg Z)
		\]
		
		

\subsection{Beweisformen}

\subsubsection{Vollständige Induktion}
	Die vollständige Induktion beweist eine Aussage für alle $n \in \mathbb{N}$ und wird vor allem bei Aussagen mit der Summenformel $\sum$ oder Teilbarkeitsaufgaben $|$ verwendet.

	\paragraph{Axiome}
	\begin{itemize}
		\item Jede Zahl aus $\mathbb{N}$ hat genau einen Nachfolger
		\item Null ist kein Nachfolger einer Zahl aus $\mathbb{N}$
	\end{itemize}

	\paragraph{Vorgehen}
	\begin{enumerate}
		\item Verankerung: Prüfen der Annahme für das kleinste mögliche $n$
		\item Induktionsschritt: $n \to n+1$
		\begin{enumerate}
			\item Annahme $A(n)$
			\item Behauptung $A(n+1)$
			\item Beweis $A(n) \Rightarrow A(n+1)$
		\end{enumerate}
	\end{enumerate}
	
	\paragraph{Beispiel}
	\[
		\text{Für } a_n = 5^n -1 \text{ mit } n \in \mathbb{N} \text{ gilt: } 4a_n
	\]
	
	\begin{enumerate}
		\item Verankerung: $n=0 \rightarrow 5^0 -1 = 0$ ist ohne Rest durch 4 Teilbar.
		\item Induktionsschritt: $n \to n+1$
		\begin{enumerate}
			\item Annahme $A(n)$: $4|5^n -1$
			\item Behauptung $A(n+1)$: $4|5^{n+1} - 1$
			\item Beweis:\newline $
				5^{n+1} -1 = 5 \cdot 5^n -1 =\newline (4 + 1) \cdot 5^n -1 = \underbracket{
					\overbracket{
						4 \cdot 5^n
					}^{\text{durch 4 Teilbar}} + \overbracket{
						5^n -1
					}^{\text{Annahme}}
				}_{\text{Induktionsbehauptung}}
			$
		\end{enumerate}
	\end{enumerate}
	
\subsubsection{Rekursion}
%TODO!!!


\end{multicols}



\section{Mengen}

\subsection{Darstellungen}
	\begin{tabular}{l l l}	
		Aufzählende Form & $\{...\}$ & $A = \{a,b,c\}$ \\
		Beschreibende Form & $\{e \in M | A(e)\}$ & $B = \{x \in A | x<3 \}$ \\
	\end{tabular}

\subsection{Operatoren}
	\begin{tabular}{l l l}
		Teilmenge & $\subset$ & $B \subset A$ und immer $\emptyset \subset M$ \\
		Vereinigung & $\cup$ & $A \cup B = \{x \in A,B | x \in A \lor x \in B\}$ \\
		Durchschnitt & $\cap$ & $A \cap B = \{x \in A | x \in B\}$ \\
		Komplement & $\overline{M}$ & $\overline{A} = \{x \in \mathbb{G} | x \not\in A\}$ \\
		Differenz & $\setminus$ & $A \setminus B = A \cap \overline{B}$  \\
		Potenzmenge & $P(M)$ & $P(A) = \{\emptyset, \{a\}, \{b\}, \{c\}, \{a,b\}, \{a,c\},\{c,b\}, A\}$ \\
		& & $|P(A)|=2^{|A|} = 8$ \\
		Kartesisches Produkt & $\times$ & $\mathbb{R} \times \mathbb{R} = \left\{(x,y) | x \in \mathbb{R}, y \in \mathbb{R} \right\}$ \\
		Mächtigkeit & $|M|$ & $|A| = 3$
	\end{tabular}

\subsection{Gesetze}
	\begin{tabular}{l l}
		Kommutativ & $A \cap B = B \cap A$ sowie $A \cup B = B \cup A$ \\
		Assoziativ & $A \cap B \cap C = (A \cap B) \cap C$ (gilt auch für $\cup$) \\
		Distributiv & $A \cup (B \cap C) = ( A \cup B) \cap (A \cup C)$ (auch für $\cup / \cap$)\\
		Idempotenz & $A \cup A = A$ und $A \cap A = A$ \\
		Verschmelzung & $A \cup (A \cap B) = A \cap (A \cup B) = A$
	\end{tabular}

\subsection{Spezialmengen}
	\begin{tabular}{l l}
		Leere Menge & $\emptyset = \{\}$ und $|\emptyset| = 0$
	\end{tabular}

\subsubsection{Relation}
	Relation $R_b$ wird zum Beispiel so angewendet:
	\[
		R_< \subset \mathbb{N} \times \mathbb{N} =
			\left\{ (x,y) \in \mathbb{N} \times \mathbb{N} | x<y \right\} =
			\left\{ (0,1), ... \right\}
	\]
	 %TODO: Ist das jetzt eine eigene Menge oder ein Operator?


\begin{multicols}{2}[
	\section{Modulo-Rechnen}
]

\subsection{Relation}
	Realtion: $y,x \in \mathbb{Z}$, $A(x,y)$, $y = y \mod z$
	\begin{align*}
		M_2^1 &= \{1,-1,3,-3,5,-5,...\} = [1] \\
		M_2^0 &= \{0,2,-2,4,-2,...\} = [0]
	\end{align*}
	
	\begin{tabular}{l|r r}
		$+$ & $[0]$ & $[1]$  \\ \hline
		$[0]$ & $[0]$ & $[1]$ \\
		$[1]$ & $[1]$ & $[0]$
	\end{tabular}
	

\subsection{Teiler-Relation}
	Wenn $a \in \mathbb{Z}, b \in \mathbb{Z}$, gilt
	\[
		T(b,a) \Leftrightarrow \exists q \in \mathbb{Z}: b \cdot q = a
	\]
	\[ T \subset \mathbb{Z} \times \mathbb{Z} \]
	\[
		\underbracket{T(b,a) \Leftrightarrow b|a}_{\text{Teiler-Relation}}
	\]
	
	Die Teiler-Relation ist eine Ordnungsrelation, d.h. eine Relation auf die menge $A, R \subset A \times A$ und ist entsprechend reflexiv, antisymetrisch und transitiv.

	Antisymetrisch heisst: $R \subset A \times A$
	\[
		\forall a, b \in A: a R b \land b R a \Rightarrow a=b
	\]

\subsection{Modulo-Operation $\mod$ / Kongruenz $\equiv$}
	Wenn $r$ und $a$ eine Zahl aus der Restklasse von $q$ ist, gilt:
	\[
		a \equiv r \mod q \Leftrightarrow q | a-r
	\]
	Zum Beispiel: $7 \equiv 1 \mod 3$

	\paragraph{Rechenregeln}
	
	Alles was für ganze Zahlen gilt, gilt auch für Modulo-Rechnung.
	
	\begin{align*}
		a &\equiv r_a \mod q \\
		b &\equiv r_b \mod q \\ \\
		a+b &\equiv r_a + r_b \mod q \\
		a \cdot b &\equiv r_a \cdot r_b \mod q \\ \\
		a^n \cdot a ^ m &= a^{n+m} \\
		(a^n)^m &= a^{n \cdot m}
	\end{align*}

\subsection{Euklidscher Algorythmus}

\subsubsection{Beispiel einfacher E.A.}
	\begin{align*}
		q &= x \text{ div } y \\
		r &= x - q \cdot y
	\end{align*}

	\begin{tabular}{l | r r r r}
		Ablauf & x & y & q & r \\ \hline
		Init. & 122 & 72 & 1 & 50 \\
		1. & 72 & 50 & 1 & 22 \\
		2. & 50 & 22 & 2 & 6 \\
		3. & 22 & 6 & 3 & 4 \\
		4. & 6 & 4 & 1 & 2 \\
		5. & 4 & $\underbracket{2}_{\mathclap{\text{ggT(122,72)}}}$ & 2 & $\underbracket{0}_{\mathclap{\text{Ende}}}$
	\end{tabular}

	Der Euklidscher Algorithmus:
	\begin{itemize}
		\item bricht immer nach n Iterationen ab 
		\item liefert den ggT(a,b)
		\item liefert das ''Inverse''
	\end{itemize}


\subsection{Beispiel Erweiterter E.A.}

	\begin{align*}
		q &= x \text{ div } y \\
		r &= x - q \cdot y \\ \\
		u &= s_{-1} \\
		s &= u_{-1} - q_{-1} \cdot s_{-1} \\
		v &= t_{-1} \\
		t &= v_{-1} - q_{-1} \cdot t_{-1}
	\end{align*}
	
	\begin{tabular}{l | r r r r r r r r }
		Ablauf & x & y & q & r & u & s & v & t \\ \hline
		Init. & 122 & 72 & 1 & 50 & 1 & 0 & 0 & 1 \\
		1. & 72 & 50 & 1 & 22 & 0 & 1 & 1 & -1 \\
		2. & 50 & 22 & 2 & 6 & 1 & -1 & -1 & 2 \\
		3. & 22 & 6 & 3 & 4 & -1 & 3 & 2 & -5 \\
		4. & 6 & 4 & 1 & 2 & 3 & -10 & -5 & 17 \\
		5. & 4 & $\underbracket{2}_{\mathclap{\text{ggT(122,72)}}}$ & 2 & $\underbracket{0}_{\mathclap{\text{Ende}}}$ &
		-10 & 
		\emph{13} & 17 &$\underbracket{-22}_{\mathllap{\text{Multipl. Inverse}}}$
	\end{tabular}
	
	\[
		ggT(x,y) = s \cdot x + t \cdot y
	\]
	
\end{multicols}
\end{document}

