% This is a default-selection of plugins that are used widely in this repo.

\documentclass[a4paper,10pt,fleqn]{article}
\usepackage[utf8]{inputenc}

% deutsche Trennmuster etc.
\usepackage[ngerman]{babel}
\usepackage[T1]{fontenc}

% mathematical simbols and fonts
\usepackage{mathtools} 
\usepackage{amssymb}
\usepackage{amsmath}
\usepackage{ntheorem}
\usepackage{polynom}
\usepackage{marvosym}
\usepackage{tabu}
\renewcommand*{\bmod}{\mathbin{\%}}
\everymath{\displaystyle}

\usepackage{multicol}
\usepackage{color}
\usepackage[usenames,dvipsnames]{xcolor}
\setlength{\columnsep}{1cm}
\setlength{\columnseprule}{0.25pt}
\def\columnseprulecolor{\color{gray}}
\usepackage{hyperref}

\usepackage[margin=1.5cm]{geometry}
\usepackage{graphicx}
\usepackage{pgfplots}
\pgfplotsset{compat=1.10}

%Code higlighting

\usepackage{minted}

% make lists more compact:
\newlength{\wideitemsep}
\setlength{\wideitemsep}{.5\itemsep}
\addtolength{\wideitemsep}{-5pt}
\let\olditem\item
\renewcommand{\item}{\setlength{\itemsep}{\wideitemsep}\olditem}
\renewcommand{\arraystretch}{1.25}

\title{Zusammenfassung DMI - Diskrete Mathematik für Informatiker}
\author{Fabian Hauser}
 
\begin{document}
\maketitle
\begin{multicols}{2}
[
\section{Logik}
]

	Bei DMI ist das Ausgedruckte Script mit Notizen zur Prüfung zugelassen.

\subsection{Operatoren}

%TODO: Hier fehlt noch was. Papier!

\subsubsection{Teiler von}
	$x$ ist Teiler von $n$: $x|n$

\subsubsection{Äquivalenz}
	Etwas ist genau gleich.
	\[
	t \Leftrightarrow t, f \Leftrightarrow f
	\]

\subsubsection{Implikation}
	Wenn a, dann b.
	
	\begin{tabular}{| c | c | c |}
	\hline 
	$A$ & $B$ & $A \Rightarrow B$ \\
	\hline
	w & w & w \\
	w & f & f \\
	f & w & w \\
	f & f & w \\
	\hline
	\end{tabular}



\subsubsection{Prädikatenlogik}
	\begin{tabular}{| l | l | l |}
		\hline
		$\forall$ & Allquantor & Für alle $x$ in $M$ gilt, \\ \hline
		$\exists$ & Existenzquantor &
		Es gibt $x$ in $M$, für \\
		& & die gilt, \\
		\hline
	\end{tabular}

	\paragraph{Beispiele}
		\begin{align*}
		A(n): & 5 | n \\
		B(n): & 1+2+3+\dots+n = \sum_{k=1}^{m}k = \frac{1}{2} n (n+1) \\
		R(x): & \text{der Weg X führt nach Rom}
		\end{align*}
		
		Alle Wege führen nach Rom:
		\[
		\forall x \in W \underbracket{:}_\text{gilt} R(x)
		\]
		
		Nicht alle Wege führen nach Rom $\Leftrightarrow$ Es gibt einen Weg, der nicht nach Rom führt:
		
		$\neg \left(\forall X \in W : R\left(X\right)\right) \Leftrightarrow \exists X \in W : \neg R(x)$
	

\subsubsection{Summenzeichen}

	$ \sum_{k=1}^{3} k = 1+2+3 = 6$

\subsection{Rechnen mit Logischen Aussagen}

Als eine \emph{Tautologie} wird eine Aussage bezeichnet, welche immer Wahr ist.

	
\subsubsection{Verknüpfungen}

	\begin{description}

		\item[Bindungsstärke] \hfill \\$\neg$ vor $\lor$, $\land$ vor $\Leftrightarrow$, $\Rightarrow$
		\item[De Morgan]\hfill \\ $\neg (A \land B) \Leftrightarrow \neg A \lor \neg B$ \newline $\neg (A \lor B) \Leftrightarrow \neg A \land \neg B$
		\item[Kommutativ] hfill \\ $A \land B \Leftrightarrow B \land A$ $A \lor B \Leftrightarrow B \lor A$
		\item[Assoziativ] \hfill \\ $A \lor (B \lor C) \Leftrightarrow (A \lor B) \lor C$
		\item[Distributiv] \hfill \\ $A \land (B \lor C) \Leftrightarrow (A \land B) \lor (A \land C)$ \newline $A \lor (B \land C) \Leftrightarrow (A \lor B) \land (A \lor C)$ 

	\end{description}


\subsubsection{Normalformen}
	Normalformen ist die Darstellung einer Logischen Aussage, wie sie aus einer Wahrheitstafel abgelesen werden kann.
	
	\paragraph{Disjunktive Normalform}
		
		Verknüpfung der logischen Ausdrücke mit $\lor$
		\[
			(X \land \neg Y \land Z) \lor
			(X \land Y \neg \land \neg Z) \lor
			(\neg X \land Y \land \neg Z) \lor
			(\neg X \land \neg Y \land \neg Z) \lor 
		\]
	
	\paragraph{Konjunktive Normalform}
		
		Verknüpfung der logischen Ausdrücke mit $\land$
		\[
			(\neg X \lor \neg Y \lor \neg Z) \land
			(\neg X \lor \neg Y \lor Z) \land
			(X \lor Y \lor \neg Z)
		\]
		
		

\subsection{Beweisformen}

\subsubsection{Direkter Beweis / Vollständige Induktion }
%TODO: Ergänzen Papier

\end{multicols}
\end{document}

