% This is a default-selection of plugins that are used widely in this repo.

\documentclass[a4paper,10pt,fleqn]{article}
\usepackage[utf8]{inputenc}

% deutsche Trennmuster etc.
\usepackage[ngerman]{babel}
\usepackage[T1]{fontenc}

% mathematical simbols and fonts
\usepackage{mathtools} 
\usepackage{amssymb}
\usepackage{amsmath}
\usepackage{ntheorem}
\usepackage{polynom}
\usepackage{marvosym}
\usepackage{tabu}
\renewcommand*{\bmod}{\mathbin{\%}}
\everymath{\displaystyle}

\usepackage{multicol}
\usepackage{color}
\usepackage[usenames,dvipsnames]{xcolor}
\setlength{\columnsep}{1cm}
\setlength{\columnseprule}{0.25pt}
\def\columnseprulecolor{\color{gray}}
\usepackage{hyperref}

\usepackage[margin=1.5cm]{geometry}
\usepackage{graphicx}
\usepackage{pgfplots}
\pgfplotsset{compat=1.10}

%Code higlighting

\usepackage{minted}

% make lists more compact:
\newlength{\wideitemsep}
\setlength{\wideitemsep}{.5\itemsep}
\addtolength{\wideitemsep}{-5pt}
\let\olditem\item
\renewcommand{\item}{\setlength{\itemsep}{\wideitemsep}\olditem}
\renewcommand{\arraystretch}{1.25}

\title{Zusammenfassung DMI - Diskrete Mathematik für Informatiker}
\author{Fabian Hauser}
 
\begin{document}
\maketitle
\begin{multicols}{2}
[
\section{Grundlagen}
]

\subsubsection{Summenzeichen}

$ \sum_{k=1}^{3} k = 1+2+3 = 6$

\subsection{Logik}

\subsubsection{Prädikatenlogik}
Beispiele:
\begin{align*}
A(n): & 5 | n \\
B(n): & 1+2+3+\dots+n = \sum_{k=1}^{m}k = \frac{1}{2} n (n+1) \\
R(x): & \text{der Weg X führt nach Rom}
\end{align*}

Dafür gibt es zwei Operatoren:

\begin{tabular}{l l l}
$\forall$ & Allquantor & Für alle $x$ in $M$ gilt, \\
$\exists$ & Existenzquantor &
Es gibt $x$ in $M$, für die gilt
\end{tabular}

Beispiel: Alle Wege führen nach Rom:
\[
\forall x \in W \underbracket{:}_\text{gilt} R(x)
\]

Nicht alle Wege führen nach Rom $\Leftrightarrow$ Es gibt einen Weg, der nach Rom führt.

$\neg \left(\forall X \in W : R\left(X\right)\right) \Leftrightarrow \exists X \in W : \neg R(x)$


%TODO: Einschub natürliche Zahlen?


\subsubsection{Direkter Beweis}
%TODO: Ergänzen Papier

\end{multicols}
\end{document}

