% This is a default-selection of plugins that are used widely in this repo.

\documentclass[a4paper,10pt,fleqn]{article}
\usepackage[utf8]{inputenc}

% deutsche Trennmuster etc.
\usepackage[ngerman]{babel}
\usepackage[T1]{fontenc}

% mathematical simbols and fonts
\usepackage{mathtools} 
\usepackage{amssymb}
\usepackage{amsmath}
\usepackage{ntheorem}
\usepackage{polynom}
\usepackage{marvosym}
\usepackage{tabu}
\renewcommand*{\bmod}{\mathbin{\%}}
\everymath{\displaystyle}

\usepackage{multicol}
\usepackage{color}
\usepackage[usenames,dvipsnames]{xcolor}
\setlength{\columnsep}{1cm}
\setlength{\columnseprule}{0.25pt}
\def\columnseprulecolor{\color{gray}}
\usepackage{hyperref}

\usepackage[margin=1.5cm]{geometry}
\usepackage{graphicx}
\usepackage{pgfplots}
\pgfplotsset{compat=1.10}

%Code higlighting

\usepackage{minted}

% make lists more compact:
\newlength{\wideitemsep}
\setlength{\wideitemsep}{.5\itemsep}
\addtolength{\wideitemsep}{-5pt}
\let\olditem\item
\renewcommand{\item}{\setlength{\itemsep}{\wideitemsep}\olditem}
\renewcommand{\arraystretch}{1.25}

\title{Zusammenfassung Dbs1 - Datenbanken 1}
\author{Fabian Hauser}
 
\begin{document}
\maketitle
\begin{multicols}{2}
[
\section{Grundlagen}
]
\subsection{Datenbankmodelle}

%TODO: Modelle!

\subsubsection{Relationale Datenbanken}


\subsubsection{Postrelationale Datenbankmodelle}
	\begin{description}
		\item[Objektrelationale Modelle] \hfill \\ Erweiterung klassische rel. DB um OO-Konzepte
		\item[Objektorientierte Modelle] \hfill \\ Modelle basierend auf OO-Sprache (z.B. java, c++)
		\item[(No|New)SQL] \hfill \\ Neuer Trend; Schema kann schnell ändern
	\end{description}

\subsubsection{ANSI-3-Ebenen-Modell}
	Das ANSI-3-Ebenen-Modell dient der Abstraktion von Daten in Ebenen, welche die Verwaltung vereinfachen soll.
	\begin{description}
		\item[externe Ebene] \hfill \\
			Objektorientiert \emph{Anwendung}
			
		\item[konzeptionelle Ebene] \hfill \\
			Relational (konzeptionelles Schema, Datenbankschema)
			
		\item[interne Ebene] \hfill \\
			Physisch. \emph{Postgresql, MySQL, etc.}
	\end{description}

\section{Konzipierung}	%TODO: Kapitulierung mit Sinn.
\subsection{UML-Diagramme}

\subsubsection{Abhängigkeiten}
\begin{description}
	\item[Komposition] \hfill \\
		Existenzielle Abhängigkeit
	\item[Aggregation] \hfill \\
		Nichtexistenzielle Abhängigkeit
\end{description}

\subsubsection{Relationen in der DB abbilden} %TODO: ???
1. Klassen 1:1 in die DB übertraen
2. In Sohnklasse übertragen
3. Alle attribute in die Vaterklases übertragen

\subsubsection{Vererbung}
\begin{description}
	\item[Disjoint] \hfill \\
	%TODO: https://en.wikipedia.org/wiki/File:Disjunkte_Mengen.svg
	\item[Incomplete] \hfill \\
	%TODO
\end{description}
	
	
\subsection{Relationales Modell}
	
	SET und ROW, keine Vererbung,
	
	Typenkonstruktionsregeln: z.B. SET OF INTEGER

\section{Relationale Algebra}

	%TODO

\subsubsection{$\Pi$ Projektion}

\subsubsection{$\Sigma$ Selection}

\subsubsection{$\multiply$ Kartesisches Produkt}

\subsubsection{$\left\|\multiply\right\|$ (inner join) / $Kreismitstrecke$ Joins }
	%TODO symbol

\subsubsection{$ \Rho $ ("repl") Unbe.}
	%TODO ?
	SIGMA
	
	X Kartesisches Produkt:
	
		Verknpüpfen aller Werte zweier Tabellen ($R1 x R2$)
	
	
	Joins: Theta-Verbdung und Equi Join
	
	
	%TODO: Überarbeiten. Siehe Dbs1 - DB-Entwurf: Rel. Modelle u. Algebra Folie 27


\subsection{Normalformen}

\subsubsection{1. Normalform}

	\begin{itemize}
		\item Attributwerte sind Atomar (d.h. einmalig)
		\item Kommalisten werden in "duplizierte" Datensätze umgewandelt
	\end{itemize}

\subsubsection{2. Normalform}

	\begin{itemize}
		\item Es wird festgelegt, welche Attribute von welcher ID \emph{funktional} abhängig sind.
		\item Nichtschlüsselattribute voll vom Schlüssel abhängig
		\item Determinante (uniq id)
	\end{itemize}

\subsubsection{3. Normalform}
	\begin{itemize}
		\item Auslagern in mehrere Tabellen mit Relationen
	\end{itemize}

\subsubsection{Boyce-Codd-Normalform}
\subsubsection{4. Normalform}
\subsubsection{5. Normalform}


\section{PGSQL}
	
	Text anhängen 
	
%TODO:	\begin{sql}
		select distinct(wohnort) || ' (Schweiz)' as "wohnort" from angestelter;
%TODO:	\end{sql}
	
	COAELESCE(vorname, '') (Nullwert ersetzten)
	
	
	GROUP BY wohnort HAVING AVG( Salaer) >= 7000
	

\subsubsection{WITH}

	WITH RECURSIVE untergebene (persnr, name, chef) AS
	(
	SELECT ...
	)
	SELECT * FROM untergebene order by chef, persnr;

\subsubsection{window functions}
	SELECT depname, empno, salary, avg(salary) OVER (PARTITION BY depname) FROM empsalary;
	
	WITH RECURSIVE untergebene (name, persnr, chef) AS
	(
		SELECT A.name, A.persnr, A.chef FROM angestellter A
			WHERE A.chef = 1010
		UNION ALL
		SELECT A.name, A.persnr, A.chef FROM angestellter A
			INNTER JOIN untergebene B ON B.persnr = A.chef
	)
	SELECT * FROM untergebene by chef, name


\subsection{Views}
	CREATE VIEW anz\_ang(anz, persnr) AS 
	  SELECT COUNT(*) as anz, persnr FROM angestellter;

\subsubsection{Materialized views}
	Der CREATE MATERIALIZED VIEW wird als Virtuelle Tabelle gespeichert. Daher muss sie auch mit REFRESH MATERIALIZED VIEW aktualisiert werden.

\subsection{Transactions}

	Keine sichtbare Parallelität, Integritätsbedingungen.
	
	Pro Connection nur eine Transaktion
	
	\subsubsection{ACID}
	\emph{A}tomicity, \emph{C}onsistency, \emph{I}solation and \emph{D}urability

	\subsubsection{SQL}
		BEGIN TRANSACTION;
		
		-- rhabarberrhabarber
		
		-- ROLLBACK TRANSACTON;
		
		COMMMIT TRANSACTION;
	\subsubsection{Serialisierbarkeit}
	Operationen werden auf lese- und schreibekonflikte (Zyklen) überprüft, und entsprechend ausgeführt.
	
	
	\subsection{Locking}
	
	- Exclusive lock: no other access possible (xlock)
	- non-exclusive lock: can be read by others. (slock)
	
	Nachdem Locks freigegeben werden, kann eine Serialisierung nicht neue Locks beantragen.
	
	
	\subsubsection{Read errors}
		READ UNCOMMITED: Alle Fehlerfälle möglich.
		
		\begin{description}
			\item[Dirty Read] \hfill \\
				Es werden Daten einer anderen Transaktion, welche Daten geändert hat, und diese gleich darauf ROLLBACKED.
				Lösung: READ COMMITTED
			\item[Fuzzy Read] \hfill \\
				In der gleichen Transaktion gibt ein SELECT-Statment verschiedene Werte aufgrund einer anderen Connection.
				Lösung: REPEATEABLE READ
			\item[Phantom Read] \hfill \\
				In der gleichen Transaktion gibt es eine andere Anzahl Tupel aus einer Tabelle.
				Lösung: SERIALIZABLE
		\end{description}
	
\section{Indexe}

B-Tree
Hash
BRIN (Block Range)
ISAM
Bitmap
GiST, GIS

Heap, 8K Speicherblöcke 

\subsection{Indextypen}
Clustered Index
Primär Index / Sekundär index (unique)
- Zusammengesetzter Index
Partitionierter Index: CREATE INDEX mytable_col_part_idx ON mytable(col) WHERE processed = 'N';
Funktionaler Index

\end{multicols}

\end{document}

