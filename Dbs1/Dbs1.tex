% This is a default-selection of plugins that are used widely in this repo.

\documentclass[a4paper,10pt,fleqn]{article}
\usepackage[utf8]{inputenc}

% deutsche Trennmuster etc.
\usepackage[ngerman]{babel}
\usepackage[T1]{fontenc}

% mathematical simbols and fonts
\usepackage{mathtools} 
\usepackage{amssymb}
\usepackage{amsmath}
\usepackage{ntheorem}
\usepackage{polynom}
\usepackage{marvosym}
\usepackage{tabu}
\renewcommand*{\bmod}{\mathbin{\%}}
\everymath{\displaystyle}

\usepackage{multicol}
\usepackage{color}
\usepackage[usenames,dvipsnames]{xcolor}
\setlength{\columnsep}{1cm}
\setlength{\columnseprule}{0.25pt}
\def\columnseprulecolor{\color{gray}}
\usepackage{hyperref}

\usepackage[margin=1.5cm]{geometry}
\usepackage{graphicx}
\usepackage{pgfplots}
\pgfplotsset{compat=1.10}

%Code higlighting

\usepackage{minted}

% make lists more compact:
\newlength{\wideitemsep}
\setlength{\wideitemsep}{.5\itemsep}
\addtolength{\wideitemsep}{-5pt}
\let\olditem\item
\renewcommand{\item}{\setlength{\itemsep}{\wideitemsep}\olditem}
\renewcommand{\arraystretch}{1.25}

\title{Zusammenfassung Dbs1 - Datenbanken 1}
\author{Fabian Hauser}
 
\begin{document}
\maketitle
\begin{multicols}{2}
[
\section{Grundlagen}
]
\subsection{Datenbankmodelle}

%TODO: Modelle!

\subsubsection{Relationale Datenbanken}


\subsubsection{Postrelationale Datenbankmodelle}
	\begin{description}
		\item[Objektrelationale Modelle] \hfill \\ Erweiterung klassische rel. DB um OO-Konzepte
		\item[Objektorientierte Modelle] \hfill \\ Modelle basierend auf OO-Sprache (z.B. java, c++)
		\item[(No|New)SQL] \hfill \\ Neuer Trend; Schema kann schnell ändern
	\end{description}

\subsubsection{ANSI-3-Ebenen-Modell}
	Das ANSI-3-Ebenen-Modell dient der Abstraktion von Daten in Ebenen, welche die Verwaltung vereinfachen soll.
	\begin{description}
		\item[externe Ebene] \hfill \\
			Objektorientiert \emph{Anwendung}
			
		\item[konzeptionelle Ebene] \hfill \\
			Relational (konzeptionelles Schema, Datenbankschema)
			
		\item[interne Ebene] \hfill \\
			Physisch. \emph{Postgresql, MySQL, etc.}
	\end{description}

\section{Konzipierung}	%TODO: Kapitulierung mit Sinn.
\subsection{UML-Diagramme}

\subsubsection{Relationen}
\begin{description}
	\item[Komposition] \hfill \\
		Existenzielle Abhängigkeit
	\item[Aggregation] \hfill \\
		
\end{description}

\end{multicols}
\end{document}

