% This is a default-selection of plugins that are used widely in this repo.

\documentclass[a4paper,10pt,fleqn]{article}
\usepackage[utf8]{inputenc}

% deutsche Trennmuster etc.
\usepackage[ngerman]{babel}
\usepackage[T1]{fontenc}

% mathematical simbols and fonts
\usepackage{mathtools} 
\usepackage{amssymb}
\usepackage{amsmath}
\usepackage{ntheorem}
\usepackage{polynom}
\usepackage{marvosym}
\usepackage{tabu}
\renewcommand*{\bmod}{\mathbin{\%}}
\everymath{\displaystyle}

\usepackage{multicol}
\usepackage{color}
\usepackage[usenames,dvipsnames]{xcolor}
\setlength{\columnsep}{1cm}
\setlength{\columnseprule}{0.25pt}
\def\columnseprulecolor{\color{gray}}
\usepackage{hyperref}

\usepackage[margin=1.5cm]{geometry}
\usepackage{graphicx}
\usepackage{pgfplots}
\pgfplotsset{compat=1.10}

%Code higlighting

\usepackage{minted}

% make lists more compact:
\newlength{\wideitemsep}
\setlength{\wideitemsep}{.5\itemsep}
\addtolength{\wideitemsep}{-5pt}
\let\olditem\item
\renewcommand{\item}{\setlength{\itemsep}{\wideitemsep}\olditem}
\renewcommand{\arraystretch}{1.25}

\title{Zusammenfassung OO - Objektorientierte Programmierung mit Java}
\author{Fabian Hauser}
 
\begin{document}
\maketitle
\begin{multicols}{2}
[
\section{Java}
]

\subsection{Variablen}

	%TODO: - wenn variable in klasse, im heap memory
	%TODO: - wenn variable in methode, im stack memory
	%TODO: - type casting
	%TODO: - heap / stack
	%TODO: - final keyword (einmal zuweisung in constructor)

\subsubsection{Typen}
	Java unterscheidet zwischen Grundlegenden Typen und erstellten Typen ...
	%TODO: - Variabletypes

\subsubsection{Strings}
	Vergleichen: $a.equals(b)$ (ausser bei String Pooling (Optimierung von Compiler bei gleichen Strings)
	Kopieren: $String b = new string(a);$

\subsection{Methoden}
	Eine Methode kann immer mit einer gleichnameigen Methode mit einer anderen Signatur überschrieben werden.
	%TODO: Demo 

\subsection{Objekte}

\subsubsection{Polymorphism}
	Eine unterklasse kann auch immer als Objekt vom übergeordneten Typ behandelt werden.
	%TODO: Demo

	Eine überklasse kann als Objekt von einem Untergeordneten Typ behandelt werden, sofern dies explizit mit einem Cast vorgegeben wurde.
	Dies funktioniert nur, wenn im Objekt von einem Untergeordneten Typ in wirklichkeit ein Objekt vom bergeordneten Objekt steckt.

	Dies funktioniert nicht mit Eigenschaften. %TODO: Stimmt das so? innerhalb der subklasse auf jeden fall. (Folie 62 OO HS2015-Vererbung)
	%TODO: Demo


\subsection{Exceptions}

\subsubsection{Beispiele}
	
	\begin{java}	
		ArayIndexOutOfBoundsExceptionArithmeticException
		NullPointerError
		StackOverflowError
		ClassCastException
		
		InputMismatchExcelption
		IllegalArgumentException
	\end{java}
	
	
	\subsubsection{Multicatch}
	
	try \{...\} catch(NoStringException e | AnotherStringException e)\{...\} finally \{ This get's executed anyways. \}

	\subsection{Packages und Import}

	1. Single Type Import $import pack.A$
	2. Eigenes Paket $class A$
	3. Import on Demand $import pack.*$


	\subsubsection{Statisch}
		import static java.lang.System.out;
		out.println("works");

\subsection{Spezielle implementationen}

\subsubsection{a.equals(b)}
... div. cases 00 HS2015 - Unit Testing Folie ~50

\subsubsection{a.clone()}

	Person implements Cloneable {

		@Override
		public Person cone() {
			return new Person(firstName, lastName);
		}
	}

\subsubsection{Comparable interface}

	int compareTo(Type other);
	Resultate: 
	<0: this is smaller than other
	>0: this is larger than other
	0: this is equal to other


	Beispiel:

	Class X implements Comparable<Student> {
		private int regNumber;
		@Override
		public int compareTo(Student other) {
			return regNumber - other.regNumber
		}
	}
\end{multicols}


\section{Quotes}
	...manchmal werden wir von den C-Leuten ein wenig gebrainwashed!
	- zur Diskussion des Unterschieds zwischen Pointern und Referenzen

\end{document}

