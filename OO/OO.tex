% This is a default-selection of plugins that are used widely in this repo.

\documentclass[a4paper,10pt,fleqn]{article}
\usepackage[utf8]{inputenc}

% deutsche Trennmuster etc.
\usepackage[ngerman]{babel}
\usepackage[T1]{fontenc}

% mathematical simbols and fonts
\usepackage{mathtools} 
\usepackage{amssymb}
\usepackage{amsmath}
\usepackage{ntheorem}
\usepackage{polynom}
\usepackage{marvosym}
\usepackage{tabu}
\renewcommand*{\bmod}{\mathbin{\%}}
\everymath{\displaystyle}

\usepackage{multicol}
\usepackage{color}
\usepackage[usenames,dvipsnames]{xcolor}
\setlength{\columnsep}{1cm}
\setlength{\columnseprule}{0.25pt}
\def\columnseprulecolor{\color{gray}}
\usepackage{hyperref}

\usepackage[margin=1.5cm]{geometry}
\usepackage{graphicx}
\usepackage{pgfplots}
\pgfplotsset{compat=1.10}

%Code higlighting

\usepackage{minted}

% make lists more compact:
\newlength{\wideitemsep}
\setlength{\wideitemsep}{.5\itemsep}
\addtolength{\wideitemsep}{-5pt}
\let\olditem\item
\renewcommand{\item}{\setlength{\itemsep}{\wideitemsep}\olditem}
\renewcommand{\arraystretch}{1.25}

\title{Zusammenfassung OO - Objektorientierte Programmierung mit Java}
\author{Fabian Hauser}
 
\begin{document}
\maketitle
\begin{multicols}{2}
[
\section{Java}
]

\subsection{Variablen}

	%TODO: - wenn variable in klasse, im heap memory
	%TODO: - wenn variable in methode, im stack memory
	%TODO: - type casting
	%TODO: - heap / stack
	%TODO: - final keyword (einmal zuweisung in constructor)

\subsubsection{Typen}
	Java unterscheidet zwischen Grundlegenden Typen und erstellten Typen ...
	%TODO: - Variabletypes

\subsubsection{Strings}
	Vergleichen: $a.equals(b)$ (ausser bei String Pooling (Optimierung von Compiler bei gleichen Strings)
	Kopieren: $String b = new string(a);$

\subsection{Methoden}
	Eine Methode kann immer mit einer gleichnameigen Methode mit einer anderen Signatur überschrieben werden.
	%TODO: Demo 

\subsection{Objekte}

\subsubsection{Polymorphism}
	Eine unterklasse kann auch immer als Objekt vom übergeordneten Typ behandelt werden.
	%TODO: Demo

	Eine überklasse kann als Objekt von einem Untergeordneten Typ behandelt werden, sofern dies explizit mit einem Cast vorgegeben wurde.
	Dies funktioniert nur, wenn im Objekt von einem Untergeordneten Typ in wirklichkeit ein Objekt vom bergeordneten Objekt steckt.

	Dies funktioniert nicht mit Eigenschaften. %TODO: Stimmt das so? innerhalb der subklasse auf jeden fall. (Folie 62 OO HS2015-Vererbung)
	%TODO: Demo


\subsection{Exceptions}

\subsubsection{Beispiele}
	
	\begin{java}	
		ArayIndexOutOfBoundsExceptionArithmeticException
		NullPointerError
		StackOverflowError
		ClassCastException
		
		InputMismatchExcelption
		IllegalArgumentException
	\end{java}
	
	
	\subsubsection{Multicatch}
	
	try \{...\} catch(NoStringException e | AnotherStringException e)\{...\} finally \{ This get's executed anyways. \}

	\subsection{Packages und Import}

	1. Single Type Import $import pack.A$
	2. Eigenes Paket $class A$
	3. Import on Demand $import pack.*$


	\subsubsection{Statisch}
		import static java.lang.System.out;
		out.println("works");

\subsection{Spezielle implementationen}

\subsubsection{a.equals(b)}
... div. cases 00 HS2015 - Unit Testing Folie ~50

\subsubsection{a.clone()}

	Person implements Cloneable {

		@Override
		public Person cone() {
			return new Person(firstName, lastName);
		}
	}

\subsubsection{Comparable interface}

	int compareTo(Type other);
	Resultate: 
	<0: this is smaller than other
	>0: this is larger than other
	0: this is equal to other


	Beispiel:

	Class X implements Comparable<Student> {
		private int regNumber;
		@Override
		public int compareTo(Student other) {
			return regNumber - other.regNumber
		}
	}
	
	\paragraph{Comparator-Object}
	
	Collections.sort(List<T> list, Comparator<T> comparator);
	
	interface Comparator<T> \{
		int compare(T first, T second);
	\}
	
\subsubsection{Wrapping von Primitive Datatypes}
	z.B. int in Integer-Objekt. Dies wird als \emph{Boxing} bzw. im umgekehrten Weg als \emph{Unboxing} bezeichnet.
\end{multicols}

\subsubsection{Array-Lists}

	ArrayList<String> stringList = new ArrayList<String>();
	.add("OO");
	.add(0, "Quark");
	.get(0);
	.remove(0); // new indexes!
	.remove("CN1"); // equals
	.contains("CN1"); // equals
	.set(0,"ICTh");
	.size();

	// Is Iterable:
	for (Strings s : stringList) {
		// s = String an pos x
	}
	
	ArrayLists werden immer mit Faktor 1.5 vergrössert.
	
\subsubsection{Linked-Lists}
	
	Pro Objekt wird ein Hilfsobjekt eingefügt, welches vor- und nachgänger kennt.
	LinkedLists sind schnell beim Einfügen, ansonsten eher langsam (ausser an enden).

\subsubsection{Set}

TreeSet
	Elemente müssen Comparable und equals() implementieren
HashSet
	Elemente müssen hashCode() und equals() implementieren

\subsubsection{Maps}

	Einseitige Assoziation z.B. Matrikelnummer -> Student

\section{Quotes}
	...manchmal werden wir von den C-Leuten ein wenig gebrainwashed!
	- zur Diskussion des Unterschieds zwischen Pointern und Referenzen

\subsubsection{Hashing / Hashmaps}
	



\subsubsection{Generics}

	class Node<t> {
		
		private T object;
		setObject(T obj) {
			object = obj;
		}
	
	
	class Node<T extends Number \& Integer> { ... }
	
	\paragraph{Wildcard-Type}
	
	Node<?> undefinedNode;
	undefinedNode = new Node(Integer)(4);
	
	

\subsection{Lambdas}

\subsubsection{Mathodenreferenzen}
	people.sort(this::compareByAge);
	
	Reference zu Methode und das beinhlatende Objekt.
	
	Sind kompatibel zu Schnittstellen:
	
	@FunctionalInterface
	interface Comparator<T> \{
		int compare(T first, T second);
	\}

\subsubsection{Implementation Lambdas}

	people.sort((Person p1, Person p2) -> {
		return p1.getAge() - p2.getAge();
	});
	
	
	Lambdas können auch Variablen zugewiesen werden. Z.B.:
	
	Comparator<Person> comp = (Person p1, Person p2) -> {
		return p1.getAge() - p2.getAge();
	};
	
	
	Typendeklaration ist optional. Der CodeBlock ist auch optional, wenn der Code eine Single Statement ist.
	
	Bei Lambdas können Umgebungensvariablen (Lokale Variablen aus der Aufrufumgebung herum) zugreifen. Wenn eine Variable in einem Lambda verwendet wird, wird die Übergebene Variable automatisch(!) $final$, d.h. darf im übergeordneten Programm nicht verändert werden nach der Zuweisung.

\subsubsection{Geschachtelte Klassen}
	class Polygon {
		private class Point {
			int x, y; // Nur innerhalb Polygon sichtbar
			
			public Point(int x, int y) {
				this.x = x; this.y = y;
				points.add(this); // Zugriff auf das "this" der äusseren Klasse erfolgt mit "Polygon.this"
			}
		}
		
		private List<Point> points = ...;
	}
	
	
	
	
	Polygon polygon1 = new Polygon();
	Polygon.Point point1 = polygon1.new Point(1,2);
	
\subsubsection{Anonyme innere Klasse}
 (relikt von der zeit vor lambdas)

	people.sort(
		new Comparator<Person>() {
			@Override
			public int compare(Person pi1, Person p2) {
				return p1.getAge() - p2.getAge();
			}
		}
	);
	
	------
	new Person("Hans", "Meier", 45, "Zürich) {
		@Override
		public String toString() {
			return "Hans Meier (30) ";
		}
	}
	
\section{Streams}

	(hat nichts mit IO-Streams zu tun)
	
	Machen Lazy Evaluation, d.h., die werte werden von der Terminalen Funktion abgerufen.
	
	Als Konsequenz: Gabelungen \emph{nicht} gestattet (nur linearer Stream).

	people.stream().filter(p -> p.getAge() >=18)
		.map(p -> p.getLastName())
		.sorted()
		.forEach(System.out::println);
		
		\subsection{Häufige Stream.Methoden:}
		filter(Predicate)
		map(Function)
		mapToInt(Function)
		mapToDouble(Function)
		sorted()
		distinct()
		limit(long n)
		skip(long n)
		reduce()
		
		\subsubsection{Terminaloperationen}
		forEach(Consumer)
		forEachOrdered(Consumer)
		count()
		min()
		max()
		average()
		sum()
		findAny()
		findFirst()
		reduce()
		
		
		\subsubsection{OptionalDouble}
		.empty()
		.of(double value)
		.ifPresent(Consumer)
		.orElse(double other)
		
		\subsubsection{True/False Queries}
		allMatch()
		anyMatch()
		noneMatch()
	
		\subsubsection{Unendliche Quellen}
		generate()
		iterate()
			
		
		\subsubsection{Parallelisierung}	
		stream.paralellStream().forEachOrdered()
	\subsection{Performance}
	
	Möglichst früh verkleinern z.B. mit filter()

\section{Streams}
	\begin{description}
	\item[Byte Streams] \hfill
		8-Bit, InputStream, OutputStream
	
	\item[Character Stream] \hfill
		16-Bit (UTF-16), Reader, Writer
	\end{description}
	
	\subsection{File-Input-Stream}
	
		fis.read() gibt einen byte-wert zurück, EOF = -1.
		
	\subsection{FileReader}
	
	(char / int)-wert
	
	Kodierung wählbar, sollte explizit angegeben werden.
	
	InputStream byteStream = new FileInoutStream("quotes.txt");
	Reader reader = new InputStreamReader(byteStream, "UTF-8");
	
	\subsection{Stream Processing}
	
	\subsubsection{BufferedReader}
	BufferedReader bufferedReader = new Bufferedreader(fileReader);
	bufferedReader.readLine();
	
\section{Serialisierung}

	class Person implements Serializable { ... }
	[...]
	
	ObjectOutputStream stream = new ObjectOutputStream(new FileOutputStream("serial.bin"));
	stream.writeObject(person);
	
	[...]
	Person p = (person)	stream.readObject();
	
\subsection{Regexp}
	import java.util.regex.*;
	bool matches = Pattern.matches("regex", input);
	
	
	\subsubsection{Group Matching}
	Pattern pattern = Pattern.compile("(asdf)(?:matching|ohne|group)(?<secondname>querz)");
	
	Matcher matcher = pattern.matcher(text);
	if(matcher.matches(){
		String firstPart = matcher.group(1);
		String secondPart = matcher.group("secondname");
	}
	
	\subsubsection{Reluctant Capturing}
		([0-9_]*?)_([0-9]*)
		
		
\subsection{Annotations}
	public @interface Temporary {
		String reason() default "unknown";
		// value: implizit default wenn nur 1 wert.
	}
	
	
	z.B.: @Temporary(raeson="Testing class") class TestLogger { }
	
	
	Mit Annotations kann auch der Java-Compiler erweitert werden.
	\subsubsection{Targets}
	@Traget(ElementType.TYPE) // kann nur für klassen verwendet werden
	public @interface Temporary { [...] }
	

	\subsubsection{Retention}
	Wie lange lebt die Annotation?
	@Retention(RetentionPolicy.RUNTIME)
	public @interface Temporary {
	}
	
	
	SOURCE // wird nur im Sourcecode angegeben
	CLASS // Wird in class-file geschrieben
	RUNTIME // JVM lädt Annotation

\subsection{Garbage Collection}

	Wenn neues Objekt, wird dies auf dem Heap in freien Platz eingefügt wird.
	Es kann zu externer (ausserhalb eines Objektes) Fragmentierung kommen. Die Garbage Collection der JVM kann den Heap kompaktifizieren. Die Speicheradresse kann sich dadurch ändern.
	D.h., die Adresse muss Abstrahiert werden.
	
	Es gibt in Java keine Delete-Möglichkeit, da ein Risiko auf Memory Leaks oder Dangling Pointers vorhanden ist.
	
	\subsubsection{Garbage Detection}
	
	1) Programm anhalten
	2) Mark: Objekte, welche vom Root-Set (Heap) her nicht mehr Verknüpft sind, dann werden diese als Garbage markiert.
	3) Sweep: Freigabe aller nicht markierten Projekte.


	Generationen: Young Generation wird häufiger geprüft.	
	
	\subsubsection{finalizer}
	
	@Override
	protected void finalize() {
		[...] some final actions before garbage collection
	}
	
	Risiken:
	- Resurrection
	- Keine Reihenfolge \& Ausführungsgarantie
	- Concurrency-Probleme, finalizer läuft in einem anderen Thread ab


\section{Good Design}
	
\end{document}

