% This is a default-selection of plugins that are used widely in this repo.

\documentclass[a4paper,10pt,fleqn]{article}
\usepackage[utf8]{inputenc}

% deutsche Trennmuster etc.
\usepackage[ngerman]{babel}
\usepackage[T1]{fontenc}

% mathematical simbols and fonts
\usepackage{mathtools} 
\usepackage{amssymb}
\usepackage{amsmath}
\usepackage{ntheorem}
\usepackage{polynom}
\usepackage{marvosym}
\usepackage{tabu}
\renewcommand*{\bmod}{\mathbin{\%}}
\everymath{\displaystyle}

\usepackage{multicol}
\usepackage{color}
\usepackage[usenames,dvipsnames]{xcolor}
\setlength{\columnsep}{1cm}
\setlength{\columnseprule}{0.25pt}
\def\columnseprulecolor{\color{gray}}
\usepackage{hyperref}

\usepackage[margin=1.5cm]{geometry}
\usepackage{graphicx}
\usepackage{pgfplots}
\pgfplotsset{compat=1.10}

%Code higlighting

\usepackage{minted}

% make lists more compact:
\newlength{\wideitemsep}
\setlength{\wideitemsep}{.5\itemsep}
\addtolength{\wideitemsep}{-5pt}
\let\olditem\item
\renewcommand{\item}{\setlength{\itemsep}{\wideitemsep}\olditem}
\renewcommand{\arraystretch}{1.25}

\title{Zusammenfassung CN1 - Computernetze 1}
\author{Fabian Hauser}
 
\begin{document}
\maketitle
\begin{multicols}{2}
\section{Signale}

\subsection{Grössen}
	%TODO: Tabelle unvollständig
	\begin{tabular} {l l l}
		Symboldauer & $T_s$ & [s] \\
		Kanalbandbreite & $B_k$ \\
		Kanalkapazität & $C$ \\
		Bandbreite & $B_s = \frac{T_s}{1}$ \\
		Schrittgeschwindigkeit & $R = f_s \cdot n$ & $\left[\frac{\text{Bit}}{\text{s}}\right]$\\ %TODO
		Abtastrate & $f_s$
	\end{tabular}

\subsection{Digitalisierung}
	Gemäss dem \emph{Abtasttheorem} gilt für die Rekonstruktion eines Signals: $f_s > B_s$

\subsection{Übertragungskanal}
	
	Im Übertragungskanal kommt in der Praxis immer ein Rauschen $N$ dazu. Je nach stärke des Rauschens wird ein Signal schwieriger zu digitalisieren. Ohne Rauschen wären unendlich viele Stufen = unendlich viele Bits pro Zeit zur verfügung.
	
	Die maximale Schrittgeschwindkeit gemäss Nyquist ist $R=2 \cdot B_k$ in $\left[\frac{\text{Schritte}}{\text{s}}\right]$
	
	Die Kanalkapazität $C$ wird berechnet mit:
	\[
		C = B_k \cdot \log_2{(1 + \frac{S}{N})}
	\]
	(unter der Voraussetzung $S$ Signalleistung und $N$ Rauschleistung)
	
\subsection{100BaseT Signaling}

\subsubsection{MLT3}
	Wenn ein 1 vorliegt, wird der Pegel gewechselt in der Mitte des Datenbits (gemäss der Clock).
	%TODO: Einfügen Bild Folie 50
	
	Benötigt 125 MBaud (20\% Verlust durch erhöhten Bandbreitenbedarf; dafür einfache Taktrückgewinnung).
	
\subsubsection{4b5b Codierung}
	Zuordnung von 4B-Symbole an 5B-Symbole. Die verbleibende 5b-Symbole werden für Statusangaben verwendet.
	Zur Sicherstellung des Takterhalt wird als Idle-Signal 11111 gesendet.
	Diese wird vor der Übertragung mit MLT3 angewendet

\subsection{Fiber}

\subsubsection{Ausgleichen der Signalgeschwindigkeit}
	Versuch, den inneren Kern der Glasfaser mit einem höheren Brechungsindex als den rest der Kerns zu machen, damit diese nicht "verschmiert" ankommen.

	64*8*1.25/(1'000'000'000b/s)
	
	1.25b/(1'000'000'000b/s) * 200'000'000m/s
	
	\subsubsection{Dämpfung und Dispersion}
	
	Wichtige Bereiche: 1310-1340nm, 1340-1440nm
	
	Im Glas wird bei 1400nm sehr stark gedämpft. %TODO: Grafik Folie 61 / 12.10.15 einbinden.
	Die Dämpfung bei ca. 1500nm ist am geringsten, und auf dieser Wellenlänge funktionieren die EDFA (Verstärkung des Signals) %TODO: Abkürzung ausschreiben)

\subsubsection{1000BASE-LX}
	L für Loong Wavelength
	
	1310nm Wellenlänge

\subsubsection{1000BASE-BX10}

FTTH nutzt diesen Standard, nutzt nur eine Singlemode Faser, bis 10 KM, Wellenlängemultiplex zur Richtungstrennung, downstream, 1490nm, upstream 1210nm
 
 
 \section{Internet Protocol IP}
 
 \subsection{AS}

 \subsubsection{As Typen}
 \begin{itemize}
	 Stub-AS
	 (Verbindung zu einzelnem AS)
	 
	 Multihomed AS (Verbindung zu mehreren anderen AS)
	 
	 
	 Transit AS (Verbindung zu mehreren AS und leitet Verkehr anderer AS weiter.	
 \end{itemize}
 
 \subsubsection{Network Levels}
 Tier-1-ISP Anschluss Zielserver
 Tier2-ISP Vermittlung
 TIER-3 ISP Endkundennetz
 
\end{multicols}
\end{document}

