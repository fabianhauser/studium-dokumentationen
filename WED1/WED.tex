% This is a default-selection of plugins that are used widely in this repo.

\documentclass[a4paper,10pt,fleqn]{article}
\usepackage[utf8]{inputenc}

% deutsche Trennmuster etc.
\usepackage[ngerman]{babel}
\usepackage[T1]{fontenc}

% mathematical simbols and fonts
\usepackage{mathtools} 
\usepackage{amssymb}
\usepackage{amsmath}
\usepackage{ntheorem}
\usepackage{polynom}
\usepackage{marvosym}
\usepackage{tabu}
\renewcommand*{\bmod}{\mathbin{\%}}
\everymath{\displaystyle}

\usepackage{multicol}
\usepackage{color}
\usepackage[usenames,dvipsnames]{xcolor}
\setlength{\columnsep}{1cm}
\setlength{\columnseprule}{0.25pt}
\def\columnseprulecolor{\color{gray}}
\usepackage{hyperref}

\usepackage[margin=1.5cm]{geometry}
\usepackage{graphicx}
\usepackage{pgfplots}
\pgfplotsset{compat=1.10}

%Code higlighting

\usepackage{minted}

% make lists more compact:
\newlength{\wideitemsep}
\setlength{\wideitemsep}{.5\itemsep}
\addtolength{\wideitemsep}{-5pt}
\let\olditem\item
\renewcommand{\item}{\setlength{\itemsep}{\wideitemsep}\olditem}
\renewcommand{\arraystretch}{1.25}

\title{Zusammenfassung WED1}
\author{Fabian Hauser}
 
\begin{document}
\maketitle
\section{Modulinformationen}
Prüfungsunterlagen: 1A4 Einseitig Handgeschrieben

\section{Grundlagen}

\begin{description}
	
	\item[Markup] \hfill \\
		Data + Markup = Information; Markup kann in ein anderes Format umgewandelt werden.
		
	\item[Programmiersprachen] \hfill \\
		Instruktionen, imperativ
	\item[Markupsprachen] \hfill \\
		Annotierung, deklarativ
	\item[Encoding] \hfill \\
		How to interprete the bytes of a document
\end{description}

\section{Webdesign / Usability}
\begin{multicols}{2}
\subsection{Gebrauchstauglichkeit}

\begin{itemize}
\item Effektiv
\item Effizient
\item Ansprechend
\item Fehlertolerant
\item Lernfördernd
\end{itemize}

\paragraph{Dialoggestaltung (ISO 9241-110}
\begin{itemize}
	\item Aufgabenangemessenheit
	\item Selbstbeschreibungsfähigkeit
	\item Steuerbarkeit
	\item erwartungskonformität
	\item Fehlertoleranz
	\item Individualisierbarkeit
	\item Lernförderlichkeit
\end{itemize}

\paragraph{UI nach Stone et al. 2005}

\begin{itemize}
	\item Visibility
	\item Affordance
	\item Feedback
	\item Simplicity
	\item Structure
	\item Consistency
	\item Tolerance
	\item Accessibility
\end{itemize}


\subsubsection{Design-Aspekte}
\begin{itemize}
	\item Oberfläche
	\item Raster
	\item Struktur
	\item Umfang
	\item Strategie
\end{itemize}
\end{multicols}

\subsubsection{Aufmerksamkeit Leiten}

\begin{itemize}
	\item Spaceing
	\item Symols
	\item Color
	\item Size
\end{itemize}

\section{CSS}

\subsection{Spezifizität}

Je höher die Zahl, umso höher die Priorität;

\begin{enumerate}
	\item *
	\item h1
	\item ul li
	\item a::after
	\item p:first-child
	\item a:not([href])
	\item ul.nav [href]
	\item \#author
	\item \#editor p
	\item \emph{style=''...''}
	\item !important
\end{enumerate}

\subsection{Grösseneinheiten}

\begin{description}
	\item[px] \hfill \\
		Basis-Einheit vom Browser, virtuelle Grösse
	\item[em] \hfill \\
		Relativ zur Parent-Schriftgrösse
	\item[rem] \hfill \\
		Relativ zum Root-Element
	\item[vw / vh] \hfill \\
		\%-Grösse vom Viewport
\end{description}

\section{JavaScript}

\subsection{Primitive Types}
It's shit.


\subsection{DOM}

Kommentare etc. sind auch im DOM-Tree.


\subsubsection{jQuery equivalences}

\begin{minted}{javascript}
document.querySelectorAll-> NodeList
document.querySelector(selector) - fetches the first matching node only -> Node
document.getElementById(idname) - fetches a single node by its ID name -> Node
document.getElementsByClassName(class) - fetches nodes with a specific class name-> NodeList

.createElement(tagName)-> Node
.createAttribute(attributeName) -> Node
.addEventListener("click", func(element));
window.addEventListener("load",  function (event){
	event.stopPropagation();
	event.target // This is leaf-node element, that the user klicked on.
	this // Aktuelles Element des Event Hanlers
})

\end{minted}

\paragraph{addClass}

\begin{minted}{javascript}
document.getElementById("MyElement").className = document.getElementById("MyElement").className.replace ( /(?:^|\s)MyClass(?!\S)/g , '' )


var liElement =  document.createElement("li");
[....]
df.appendChild(liElement);
\end{minted}

Links:
\begin{itemize}
	\item \url{https://developer.mozilla.org/en-US/docs/Web/API/EventTarget/addEventListener}
	\item \url{https://developer.mozilla.org/en-US/docs/Web/API/Event}
\end{itemize}

\subsection{Testat}

node.js 6.0.0



\end{document}