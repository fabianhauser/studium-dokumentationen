% This is a default-selection of plugins that are used widely in this repo.

\documentclass[a4paper,10pt,fleqn]{article}
\usepackage[utf8]{inputenc}

% deutsche Trennmuster etc.
\usepackage[ngerman]{babel}
\usepackage[T1]{fontenc}

% mathematical simbols and fonts
\usepackage{mathtools} 
\usepackage{amssymb}
\usepackage{amsmath}
\usepackage{ntheorem}
\usepackage{polynom}
\usepackage{marvosym}
\usepackage{tabu}
\renewcommand*{\bmod}{\mathbin{\%}}
\everymath{\displaystyle}

\usepackage{multicol}
\usepackage{color}
\usepackage[usenames,dvipsnames]{xcolor}
\setlength{\columnsep}{1cm}
\setlength{\columnseprule}{0.25pt}
\def\columnseprulecolor{\color{gray}}
\usepackage{hyperref}

\usepackage[margin=1.5cm]{geometry}
\usepackage{graphicx}
\usepackage{pgfplots}
\pgfplotsset{compat=1.10}

%Code higlighting

\usepackage{minted}

% make lists more compact:
\newlength{\wideitemsep}
\setlength{\wideitemsep}{.5\itemsep}
\addtolength{\wideitemsep}{-5pt}
\let\olditem\item
\renewcommand{\item}{\setlength{\itemsep}{\wideitemsep}\olditem}
\renewcommand{\arraystretch}{1.25}

\title{Zusammenfassung ICTh - Informations- und Codierungstheorie}
\author{Fabian Hauser}
 
\begin{document}
\maketitle
\begin{multicols}{2}
[
\section{Signale}
]

\subsection{Periodische Signale (s.1-16)}

\subsubsection{Charakteristische Grössen}
\begin{tabular}{l l l}
	Periodendauer & $\top$ & [s] \\
	 & $s(t)$ & $=s(t+mT)$ \\
	 & & $m=0,\pm 1,\pm 2,\dots$ \\
	 Grundfrequenz & $f_0=\frac{1}{\top}$ & [Hz]
\end{tabular}

Jedes periodische Signal kann in seine harmonische Komponenten zerlegt werden (Fourier-Analyse, FFT)

\subsubsection{Sinusfunktion}
	\begin{align*}
	s(t) &= A \sin{ 2 \pi \frac{t}{\top}} = A \sin{2 \pi f_0 t} \\
	s(\varphi) &= A \sin{\varphi} \\
	0&\leq \varphi \le 2 \pi
	\end{align*}
	
	\noindent\begin{tikzpicture}
	\begin{axis}[
		axis equal image,
		% x-Achse
		axis x line=center,
		xlabel={$t \left[\text{s}\right]$},
		xtick={1.5708,3.14159,4.7123889, 6.283185},
		xticklabels={$\frac{\top}{4}$,$\frac{\top}{2}$,$\frac{3\top}{4}$, $\top$},
		% y-achse
		height= 6cm,
		axis y line = left,
		ytick={-1,0,1},
		yticklabels={$-A$,$0$,$+A$},
		ylabel={$s(t) \left[\text{???}\right]$} %TODO: Skala?
	]
	\addplot [color=blue,domain=0:2.07*pi,samples=100] {sin(deg(x))};
	\end{axis}
	\end{tikzpicture}


\subsubsection{Cosinusfunkton}
	$s(t) = A \cos{2 \pi \frac{t}{\top}} = A \cos{2 \pi f_0 t}$
	
%TODO: Graph

	
	
	$s(\varphi) = A \cos{\varphi}$
	$0 \leq \varphi \le 2 \pi$
	

\subsubsection{Linearer Mittelwert}

	In der Praxis sollte der lineare Mittelwert 0 sein, um die Messung zu vereinfachen.

	\begin{align*}
		A_0 &= \frac{\tau}{T} A = \frac{1}{T} \int\limits_{t_0}^{t_0 + T}{s(t) dt} \\
		\to A_0 T &=  \int\limits_{t_0}^{t_0 + T}{s(t) dt}
	\end{align*}

\subsubsection{Fourier-Reihe}

	Mittels der Fourier-Reihe kann jedes Signal mittels einer unendlichen Summe von Sinus- und Cosinusfunktionen optimal angenähert werden.

	$s_K(t) = A_0 + \sum^k_{k=1}{A_k \cos{(2 \pi k \frac{t}{T})} + B_k \sin{(2 \pi k \frac{t}{T})}}$

	%TODO: Folie 11 (Optimale Signalapproximation mit kleinstem quadratischen Fehler)

\subsubsection{Sinusapproximation}
	%TODO: $\otimes$ => Multiplikation zweier analoger Signale
	%TODO: Folie 13,15

\end{multicols}
\end{document}

