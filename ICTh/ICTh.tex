% This is a default-selection of plugins that are used widely in this repo.

\documentclass[a4paper,10pt,fleqn]{article}
\usepackage[utf8]{inputenc}

% deutsche Trennmuster etc.
\usepackage[ngerman]{babel}
\usepackage[T1]{fontenc}

% mathematical simbols and fonts
\usepackage{mathtools} 
\usepackage{amssymb}
\usepackage{amsmath}
\usepackage{ntheorem}
\usepackage{polynom}
\usepackage{marvosym}
\usepackage{tabu}
\renewcommand*{\bmod}{\mathbin{\%}}
\everymath{\displaystyle}

\usepackage{multicol}
\usepackage{color}
\usepackage[usenames,dvipsnames]{xcolor}
\setlength{\columnsep}{1cm}
\setlength{\columnseprule}{0.25pt}
\def\columnseprulecolor{\color{gray}}
\usepackage{hyperref}

\usepackage[margin=1.5cm]{geometry}
\usepackage{graphicx}
\usepackage{pgfplots}
\pgfplotsset{compat=1.10}

%Code higlighting

\usepackage{minted}

% make lists more compact:
\newlength{\wideitemsep}
\setlength{\wideitemsep}{.5\itemsep}
\addtolength{\wideitemsep}{-5pt}
\let\olditem\item
\renewcommand{\item}{\setlength{\itemsep}{\wideitemsep}\olditem}
\renewcommand{\arraystretch}{1.25}

\title{Zusammenfassung ICTh - Informations- und Codierungstheorie}
\author{Fabian Hauser}
 
\begin{document}
\maketitle
\begin{multicols}{2}
[
\section{Signale}
]

\subsection{Periodische Signale}

\subsubsection{Charakteristische Grössen}
\begin{tabular}{l l l}
	Periodendauer & $\top$ & [s] \\
	 & $s(t)$ & $=s(t+mT)$ \\
	 & & $m=0,\pm 1,\pm 2,\dots$ \\
	 Grundfrequenz & $f_0=\frac{1}{\top}$ & [Hz] \\
\end{tabular}

Jedes periodische Signal kann in seine harmonische Komponenten zerlegt werden (Fourier-Analyse, FFT)

\subsubsection{Sinusfunktion}
	\begin{align*}
	s(t) &= A \sin{ 2 \pi \frac{t}{\top}} = A \sin{2 \pi f_0 t} \\
	s(\varphi) &= A \sin{\varphi} \\
	0&\leq \varphi \le 2 \pi
	\end{align*}
	
	\noindent\begin{tikzpicture}
	\begin{axis}[
		axis equal image,
		% x-Achse
		axis x line=center,
		xlabel={$t \left[\text{s}\right]$},
		xtick={1.5708,3.14159,4.7123889, 6.283185},
		xticklabels={$\frac{\top}{4}$,$\frac{\top}{2}$,$\frac{3\top}{4}$, $\top$},
		% y-achse
		height= 6cm,
		axis y line = left,
		ytick={-1,0,1},
		yticklabels={$-A$,$0$,$+A$},
		ylabel={$s(t) \left[\text{(Volt)}\right]$}
	]
	\addplot [color=blue,domain=0:2.07*pi,samples=100] {sin(deg(x))};
	\end{axis}
	\end{tikzpicture}


\subsubsection{Cosinusfunkton}

	\begin{align*}
		s(t) &= A \cos{2 \pi \frac{t}{\top}} = A \cos{2 \pi f_0 t} \\
		s(\varphi) &= A \cos{\varphi} \\
		0 &\leq \varphi \le 2 \pi
	\end{align*}
	
	\noindent\begin{tikzpicture}
	\begin{axis}[
	axis equal image,
	% x-Achse
	axis x line=center,
	xlabel={$t \left[\text{s}\right]$},
	xtick={1.5708,3.14159,4.7123889, 6.283185},
	xticklabels={$\frac{\top}{4}$,$\frac{\top}{2}$,$\frac{3\top}{4}$, $\top$},
	% y-achse
	height= 6cm,
	axis y line = left,
	ytick={-1,0,1},
	yticklabels={$-A$,$0$,$+A$},
	ylabel={$s(t) \left[\text{(Volt)}\right]$}
	]
	\addplot [color=blue,domain=0:2.07*pi,samples=100] {cos(deg(x))};
	\end{axis}
	\end{tikzpicture}

	

\subsubsection{Linearer Mittelwert}

	In der Praxis sollte der lineare Mittelwert 0 sein, um die Messung zu vereinfachen.

	\begin{align*}
		A_0 &= \frac{\tau}{T} A = \frac{1}{T} \int\limits_{t_0}^{t_0 + T}{s(t) dt} \\
		\to A_0 T &=  \int\limits_{t_0}^{t_0 + T}{s(t) dt}
	\end{align*}
	
\subsubsection{Mittlere Leistung}

\[
P = A_0^2 + \sum^{\infty}_{k=1}{
		\frac{A_k^2 + B_k^2}{2}
	} = A_0^2 + \sum^{\infty}_{k=1}{\frac{M_k^2}{2}}
\]

\emph{Sie können ruhig weiterschlafen, wir werden damit nicht sehr viel machen.}

\subsubsection{Fourier-Reihe}

	Mittels der Fourier-Reihe kann jedes periodische Signal anhand einer unendlichen Summe von Sinus- und Cosinusfunktionen optimal angenähert werden.

	$s_K(t) = A_0 + \sum^k_{k=1}{A_k \cos{(2 \pi k \frac{t}{T})} + B_k \sin{(2 \pi k \frac{t}{T})}}$

	%TODO: Folie 11 (Optimale Signalapproximation mit kleinstem quadratischen Fehler)

\subsubsection{Sinusapproximation}
	%TODO: $\otimes$ => Multiplikation zweier analoger Signale
	%TODO: Folie 13,15

\subsubsection{Funktionseigenschaften}

\paragraph{Gerade Funktion}
	\[
		S(t) = S(-t)
	\]
	Achsensymetrisch (gespiegelta an der Y-Achse)
	
	Dies trifft auf den Cosinus zu.

\paragraph{Ungerade Funktion}
	\[
		S(t) = -S(-t)
	\]
	Punktsymetrisch (gespiegelt am Achsenschnittpunkt)
	
	Dies trifft auf den Sinus zu.
	
	Nulldurchgänge:
	\[
		\pi \cdot \tau \cdot f_k = n \pi \Rightarrow  f_k = \frac{n}{\tau}
	\]

Frequenzspektrum:
	\[
		\left| S(t) \right| = \left| \frac{\sin{\pi \tau f}}{\pi \tau f} \right|
	\]		


\subsection{Fourier-Transformation}

\subsubsection{Grössen}
	\begin{tabular}{l l l}
		Amplitudendichte & & $\left[\frac{\text{V}}{\text{Hz}}\right]$
	\end{tabular}

\subsubsection{Einzelpuls}
	Ein einzelner Puls mit der Dauer $\top$ gegen $\infty$. Dies kann mittels der Fourier-Transformation ermittelt werden:
	
	\[
		S(t) = \int\limits_{-\infty}^{\infty}{
			S(t) \cdot e^{-j 2 \pi f_t} dt
		}
		\left[\frac{\text{V}}{\text{Hz}}\right]
	\]

\subsubsection{Energie eines Pulses}

	\[
		W = \int\limits_{-\infty}^{\infty}{
				\left| S(t) \right|^2 dt
			} = \int\limits_{-\infty}^{\infty}{
				\left| S(f) \right|^2 dt
			}
	\]

\subsubsection{Symbolrate}
	\[
		f_s = \frac{1}{\tau} = \frac{\text{Symbole}}{\text{Zeit}}
		\left[\text{Baud}\right] \left[\text{Hz}\right]
	\]

\subsubsection{Bitrate}
	\[
		f_b = f_s \cdot \frac{\text{Bit}}{\text{Symbol}} %TODO: Gibt's hierfür ein Symbol?
		\left[\frac{\text{Bit}}{\text{s}}\right]
	\]
	
\subsubsection{Bandbreite}
	\[
		B=\frac{1}{\tau}
	\]
	
	\[
		[\text{dB}] = \frac{1}{\tau} \cdot \log{\frac{2}{\pi}}
	\]

Je kürzer die Dauer im Zeitbereich, desto grösser die Bandbreite:
	\[
		B \cdot \tau \approx 1
	\]

\subsection{Nichtperiodische Signale}

\subsubsection{Charakteristische Grössen}
	
	\begin{tabular}{l l l}
		Pulsdauer & $\tau$ & [s] \\
		Pusenegerie & $W=A^2 \cdot \tau$ & \\
		Mittlere Leistung & $P=\frac{W}{T}$ &
	\end{tabular}
	\[
		\text{Zeitbereich } S(t) \mapsto \text{Frequenzbereich } S(t)
	\]


\subsubsection{Linearer Mittelwert}

	\[
		A \cdot \tau = A_0 \cdot \pi \Rightarrow A_0 =\frac{\tau}{\top} \cdot A
	\]

\subsection{Datenkodierungs}

\subsubsection{Non Return Zero (NRZ) Code}
	Als beim NRZ Code

\subsubsection{Non Return Zero (NRZ) Mark Code}
	Beim NRZ Mark Code wird der Pegel $A$ gewechselt, wenn die übertragene Zahl eine logische $1$ ist. Dies wird auch als Differenzielle Kodierung bezeichnet.

\subsubsection{Return to Zero (RZ) Code}
	Beim RZ Code wieturn Zerord in der Mitte des Datenbits der Pegel bei einem logischen $1$ wieder auf $0$ gesenkt. Dadurch braucht diese doppelt so viel Bandbreite.
	Dafür lässt sich der Übertragungstakt aus dem Signal extrahieren.

\subsubsection{Alternate Mark Inversion (AMI) Code}
	Beim AMI wird bei einem Bitwechsel die Flanke geändert. Dadurch können mt dem AMI Weniger Daten kodiert werden, dafür ist aber die Gleichspannung sichergestellt.

	%TODO: Einfügen Bild

\subsubsection{Bi-Phase oder Manchester Code}
	Beim Manchester Code werden die Datenbits in der Flanke (d.h. $1$ auf- bzw. $0$ Absteigend) codiert. Dadurch ist dieser Code Gleichspannungsfrei.
	Der von IEEE spezifizierte Manchester Code hat invertierten Flanken.
	Die Schwierigkeit beim Manchester Code ist es, die Signifikante Steigende / Fallende Flanke zu definieren.

	%TODO: Einfügen Bild präsentation!
	
\subsubsection{Scrambling}
	Zur Eliminierung von langen 0 bzw. 1-Folgen werden die übertragenen Bits via $\oplus$ (XOR) mit einer Scrambling-Folge verknüpft.
	Die Scrambling-Folge ist dabei eine pseudo-zufällig generierte Folge auf Sende- und Empfangsstelle.
	
	Dies wird bei den Nachfolgern von 10Mbit/s Ethernet mit NRZ verwendet.

\subsection{Modulation}
	
	Bei der Modulation wird die Amplitude oder die Phase verändert, um Daten auf ein Grundsignal aufzumodulieren.
	

\subsubsection{Signale}
	
	
	Gemäss Euler:
	\[
		\cos \varphi = \frac{
			e^{+j \varphi} + e^{-j \varphi}
		}{2}
	\]
	\[
		\sin \varphi = \frac{
			e^{+j \varphi} - e^{-j \varphi}
		}{2j}
	\]
	
	%TODO: {Grafik}


\subsection{Zeitliche Quantisierung}

	Bei der Abtastung wiederholt sich das Spektrum im Intervall der Abtastfrequenz.
	
	Nyquist-Theorem: Bandbreite < Frequenzspektrum(s)
	
	Vor jeder Abtastung braucht es die Beschränkung des Frequenzspektrums auf die Hälfte der Abtastfrequenz (sonst gibt es Überlagerungen des Spektrums.)

	

\end{multicols}

	\emph{Das Spektrum müssen Sie sich vorstellen, wie wenn ich am Big Bang am Zeiger hänge und den versuche zurückzudrehen. \--- So zumindest stelle ich mir das in meinen Träumen vor.}

\end{document}

