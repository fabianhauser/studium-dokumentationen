% This is a default-selection of plugins that are used widely in this repo.

\documentclass[a4paper,10pt,fleqn]{article}
\usepackage[utf8]{inputenc}

% deutsche Trennmuster etc.
\usepackage[ngerman]{babel}
\usepackage[T1]{fontenc}

% mathematical simbols and fonts
\usepackage{mathtools} 
\usepackage{amssymb}
\usepackage{amsmath}
\usepackage{ntheorem}
\usepackage{polynom}
\usepackage{marvosym}
\usepackage{tabu}
\renewcommand*{\bmod}{\mathbin{\%}}
\everymath{\displaystyle}

\usepackage{multicol}
\usepackage{color}
\usepackage[usenames,dvipsnames]{xcolor}
\setlength{\columnsep}{1cm}
\setlength{\columnseprule}{0.25pt}
\def\columnseprulecolor{\color{gray}}
\usepackage{hyperref}

\usepackage[margin=1.5cm]{geometry}
\usepackage{graphicx}
\usepackage{pgfplots}
\pgfplotsset{compat=1.10}

%Code higlighting

\usepackage{minted}

% make lists more compact:
\newlength{\wideitemsep}
\setlength{\wideitemsep}{.5\itemsep}
\addtolength{\wideitemsep}{-5pt}
\let\olditem\item
\renewcommand{\item}{\setlength{\itemsep}{\wideitemsep}\olditem}
\renewcommand{\arraystretch}{1.25}

\title{Zusammenfassung AutoSpr}
\author{Fabian Hauser}
 
\begin{document}
\maketitle

\section{Varia}
Unterrichtszeiten: 12:00-12:30, 13:45-14:15, 14:20-14:50

Prüfung: Zusammenfassung $1m^2  = 8 \cdot A4$, Taschenrechner


\section{Notation}
Prädikat:	Aussagen

\subsection{Logik}
Siehe Seite 5 Script!

\subsection{Mengenlehre}

\section{Sprachen}

\subsection{Alphabet}
Definition : Nicht leere Menge von "Zeichen".

Zum Beispiel: $\Sigma = \{0, 1\}$, $\Sigma = \{1\}$, $\Delta = \{a,b,c,...,z\}$


\subsection{Wort}

Definition: Zeichenkette von Zeichen aus $\Sigma$: $w \in \Sigma^n = \Sigma \times \Sigma \times ... \times \Sigma$. Länge: $|w| = n$
\[
	\Sigma^0 = \{\epsilon\}; \epsilon = \text{ leeres Wort}
\]
\[
	\Sigma^\ast = \Sigma^0 \cup \Sigma^1 \cup ... \cup \Sigma^n = \bigcup^{\infty}_{k=0}{\sum^k}
\]

Im Zusammenhang mit der Sprache sind Zeichenketten bedeutungslos!

\paragraph{Beispiele}

\begin{align*}
	\Sigma &= \{1\} & \Sigma^\ast &= \{\epsilon, 1,11,111,1111,...\} \\
	L &\subset \Sigma^\ast  & L &=\{1,11,1111,11111111,....\} = \{w \in \Sigma^\ast| |w|=2^k, k \in \mathbb{N}\}
\end{align*}


\subsection{Notationen}
Beispiele:
\begin{align*}
	\Sigma &= \{0,1\} \\
	L_1 &= \{0^n 1^n | n \geq 0\} &\rightarrow 0^n = 00...0 \\
	L_2 &= \left\{ w \in \Sigma^\ast \left| \left|w\right|_0 = \text{ Anzahl 0 in } w = \left|w\right|_1 \right.\right\} \\
	L_3 &= \left\{w \in \Sigma^\ast \left| \text{ Zahlenwert von } w \text{ ist durch 3 teilbar} \right.\right\}
\end{align*}

Sprache für Graphen:

\begin{align*}
V & \text{ Vertices} \\
E & \text{ Edges} &= \{a,b\}, a,b \in V \\
G &= (V, E) &= (\{0,1,9,27\},\{\{0,1\},\{1,3\},\{1,27\},...\}) \\
\Sigma &= \left\{\ (,),",","\{","\}", 0, 1,2,...,9 \right\}
\end{align*}
jeder Graph lässt sich so codieren: $g \subset \Sigma^\ast, g = \{w \in \Sigma^\ast \left| w \text{ ist eine Graph-Beschreibung} \right.\}$

\end{document}